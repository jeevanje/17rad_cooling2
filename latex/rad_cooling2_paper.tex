%% Draft settings
\documentclass[10pt]{article}
\usepackage{amsmath}
\usepackage{amssymb}
\usepackage{graphicx}
%\usepackage{subfigure}
\usepackage{subcaption}
%\usepackage{caption}
%\usepackage[countmax]{subfloat}
\usepackage{color}
 \usepackage{lineno}
\usepackage{simplemargins}
\usepackage{natbib}

% \linenumbers*[1]
 \usepackage[T1]{fontenc} % For citing barki{\dj}ija
\setkeys{Gin}{draft=false}


% Margins
\setleftmargin{1in}
\setrightmargin{1in}
\setbottommargin{1in}
\settopmargin{1in}
 

% Standard shortcuts
\input{/Users/nadir/Dropbox/resources/shortcuts.tex}

% radiation shorthand
\newcommand{\OLRk}{\ensuremath{\mathrm{OLR}_k}}
\newcommand{\trans}{\ensuremath{\mathcal{T}}}
\newcommand{\cool}{\ensuremath{\mathcal{C}}}
\newcommand{\ch}{\ensuremath{\mathcal{H}}}
\newcommand{\chk}{\ensuremath{\ch_k}}
\newcommand{\chnet}{\ensuremath{\ch_\mathrm{net}}}
\newcommand{\chcts}{\ensuremath{\mathcal{H}^\CTS}}
\newcommand{\chkcts}{\ensuremath{\ch_k^\CTS}}
\newcommand{\pierre}{P10}
\newcommand{\pem}{\ensuremath{p_1}}
\newcommand{\lk}{\ensuremath{l_k}}
\newcommand{\lj}{\ensuremath{l_j}}
\newcommand{\tauk}{\ensuremath{\tau_k}}
\newcommand{\tauks}{\ensuremath{\tau_{k,s}}}
\newcommand{\taus}{\ensuremath{\tau_s}}
\newcommand{\tautilde}{\ensuremath{\tilde{\tau}}}
\newcommand{\Bs}{\ensuremath{B_s}}
\newcommand{\CTS}{\ensuremath{\mathrm{CTS}}}
\newcommand{\mubar}{\ensuremath{\bar{\mu}}}
\newcommand{\kapparef}{\ensuremath{\kappa_{\mathrm{ref}}}}
\newcommand{\kappao}{\ensuremath{\kappa_0}}
\newcommand{\Tref}{\ensuremath{T_{\mathrm{ref}}}}
\newcommand{\pref}{\ensuremath{p_{\mathrm{ref}}}}
\newcommand{\WVP}{\ensuremath{\mathrm{WVP}}}
\newcommand{\Tav}{\ensuremath{T_{\mathrm{av}}}}
\newcommand{\Tstrat}{\ensuremath{T_{\mathrm{strat}}}}
\newcommand{\Tstar}{\ensuremath{T^*}}
\newcommand{\Tone}{\ensuremath{T_1}}
\newcommand{\Tinf}{\ensuremath{T_\infty}}
\newcommand{\kappamin}{\ensuremath{\kappa_{\mathrm{min}}}}
\newcommand{\omegadiab}{\ensuremath{\omega_{\mathrm{diabatic}}}}



% kappa param variables
\newcommand{\kapparot}{\ensuremath{\kappa_{\mathrm{rot}}}}
\newcommand{\kappavr}{\ensuremath{\kappa_{\mathrm{vr}}}}
\newcommand{\kappaQ}{\ensuremath{\kappa_Q}}
\newcommand{\krot}{\ensuremath{k_\mathrm{rot}}}
\newcommand{\kvr}{\ensuremath{k_\mathrm{vr}}}
\newcommand{\konerot}{\ensuremath{k_{1,\mathrm{rot}}}}
\newcommand{\konevr}{\ensuremath{k_{1,\mathrm{vr}}}}
\newcommand{\kQ}{\ensuremath{k_Q}}
\newcommand{\koneP}{\ensuremath{k_{1,P}}}
\newcommand{\koneR}{\ensuremath{k_{1,R}}}
\newcommand{\konej}{\ensuremath{k_{1,j}}}
\newcommand{\lrot}{\ensuremath{l_\mathrm{rot}}}
\newcommand{\lvr}{\ensuremath{l_\mathrm{vr}}}
\newcommand{\lQ}{\ensuremath{l_{Q}}}
\newcommand{\vr}{\ensuremath{\mathbf{vr}}}
\newcommand{\rot}{\ensuremath{\mathbf{rot}}}


%Variables
\newcommand{\figurepath}{../figures/}


\begin{document}

%% ------------------------------------------------------------------------ %%
%
%  TITLE
%
%% ------------------------------------------------------------------------ %%


\title{An Analytical Model for Real Gas Radiative Cooling}

%% ------------------------------------------------------------------------ %%
%
%  AUTHORS AND AFFILIATIONS
%
%% ------------------------------------------------------------------------ %%


 \author{Nadir Jeevanjee and Stephan Fueglistaler}

\maketitle

\begin{abstract}
Atmospheric radiative cooling is a fundamental aspect of the Earth's greenhouse effect and is intrinsically connected to  atmospheric motions. Clear-sky longwave cooling turns out to be remarkably uniform throughout the troposphere, taking on a characteristic value of roughly 2 K/day. Though these aspects are robustly reproduced by models, an intuitive understanding is lacking.

Here we pursue such understanding by building a simple model for real gas (rather than gray gas) clear-sky  radiative cooling. We  focus on \htwo, consider  a single idealized atmospheric column, and employ the line-by-line Reference Forward Model (RFM) for reference and validation.  We  combine the cooling-to-space approximation with simplified greenhouse gas spectroscopy and analytical expressions for optical depth to build a simple, analytical model which reproduces the characteristic cooling  of 2 K/day and as well as the relative insensitivity of radiative cooling to relative humidity and temperature. This model expresses cooling rates as a product of the Planck function, an emissivity gradient, and a characteristic spectral width  derived from our simplified spectroscopy. The simple model is then used to investigate  several key aspects of radiative cooling and outgoing longwave radiation (OLR), including the upper tropospheric kink in radiative cooling, the \Ts-invariance of \cite{jeevanjee2018},  the spectral peak in OLR, and the contrast in radiative cooling between \htwo\ and \cotwo.




%\vspace{0.5cm}
%
%
\end{abstract}


%% ------------------------------------------------------------------------ %%
%
%  TEXT
%
%% ------------------------------------------------------------------------ %%


\section {Introduction}
The emergence of a turbulent troposphere in Earth's atmosphere stems from instabilities of the pure radiative equilibrium state (PRE). These instabilities lead to deep convection in the tropics and baroclinic eddies in mid-latitudes, both of which lead to temperature profiles which deviate markedly from those of PRE.  Since PRE is by definition an atmospheric state with zero atmospheric radiative cooling, deviations from PRE inevitably lead to radiative cooling, which is thus tightly coupled to turbulence and the hydrological cycle. Furthermore, the magnitude of this radiative cooling characterizes the large-scale circulation, as clear air subsiding in an environment with lapse-rate $\Gamma \equiv - \der{T}{z}$ and clear-sky net radiative heating rate \chnet\ (K/day, positive for heating)  must subside diabatically in steady-state with pressure velocity $\omegadiab   =  - \rho g\chnet/\Gamma$.

A remarkable feature of \chnet\ and its longwave component \ch\  is that both are  surprisingly uniform throughout the troposphere,  with a typical value of $\ch\approx -2 $K/day  (Fig. \ref{ecmwf_vs_rfm}a; partial compensation by shortwave heating results in $\chnet\approx -1$ K/day, not shown). This value of \ch\ is fairly constant in both latitude and height, which means that apart from perturbations  due to clouds and water vapor variability, the clear-sky troposphere can be considered as diabatically subsiding everywhere at a fairly uniform rate of $\omegadiab\approx\ 30$ hPa/day . It thus seems worthwhile to try and understand why $\ch \approx -2$ K/day, and why this number seems so robust.

For complex systems such as Earth's climate, such an understanding must be comprised of a hierarchy of models \citep{jeevanjee2017a,held2005}. An attempt to construct such a hierarchy for radiative cooling might proceed as follows. We first collapse the cooling distribution in Fig. \ref{ecmwf_vs_rfm}a onto a global radiative cooling profile by taking a meridional average (Fig. \ref{ecmwf_vs_rfm}b, solid line). We then find empirically that this global-mean heating profile can be emulated by that generated by a single-column, line-by-line calculation of radiative cooling from \htwo\ only in an idealized\footnote{Note that this calculation has no \htwo\ continuum or \cotwo\ absorption, which are errors that compensate (section \ref{sec_cont_co2}).} tropical atmosphere using the line-by-line Reference Forward Model (RFM, dashed line of Fig. \ref{ecmwf_vs_rfm}b,  further details in Section \ref{sec_rfm_calcs}).  Though this 1D RFM calculation does a decent job of emulating the global average profile,  even this calculation is too complex to provide much understanding, as it involves the convolution of complex greenhouse gas spectroscopy with non-linear radiative transfer. Turning to simpler models, then, we calculate radiative cooling using a gray radiation model, using the same thermodynamic profiles and absorber concentration as the RFM calculation, but using a non-pressure-broadened gray absorption coefficient tuned to yield the same OLR as the RFM calculation. This gray radiation model turns out to be \emph{too} simple, however;  its radiative cooling profile has too much vertical structure, and its lower-tropospheric heating rates err by a factor of two to three (Fig. \ref{ecmwf_vs_rfm}b, gray line). There is thus a gap between our simulation and understanding of radiative cooling, and it is the goal of this paper to help fill that gap.

To reduce the complexity of `real gas' (as opposed to gray gas) radiative cooling, we will simplify the radiative transfer using  the cooling-to-space approximation \citep[e.g.][]{petty2006}, and simplify the spectroscopy by approximating the spectrum of mass absorption coefficients $\kappa(k)$ as piecewise exponentials, 
% justify this theoretically with references?
as also done in \cite{wilson2012}. These simplifications not only make the real gas radiative transfer analytically tractable, but also lead us to a characteristic spectral width across which any absorption band of a given gas will emit at a given height. This characteristic width provides a crucial  ingredient in understanding \ch, which we can then express  as the product of a Planck function times an optical depth gradient times this width. The resulting \ch\ profiles successfully emulate those produced by RFM, and can then be used to understand the magnitude of \ch\ as well as its  (in)sensitivities to temperature and humidity. We then apply our formalism  to understand the upper tropospheric kink in radiative cooling, the \Ts-invariance of \cite{jeevanjee2018},  the spectral peak in OLR, and the contrast in radiative cooling between \htwo\ and \cotwo. Only clear-sky radiative cooling will be considered here.

\section{Preliminaries}
%RFM_calcs
\subsection{RFM calculations} \label{sec_rfm_calcs}
We use the Reference Forward Model \citep[RFM,][]{dudhia2017}, a line-by-line, longwave radiative transfer model. We use HITRAN spectroscopic data for the most common isotopologue of \htwo\ from 0--1500 \cminverse, and for our base case consider an idealized atmosphere with $\Ts=300\ \Kelvin$,  a constant lapse rate of $\Gamma= 7\ \Kelvin/\km$ up to to an isothermal stratosphere at $\Tstrat=200$ K, and \RH=0.75 (except where specified). We run RFM at a spectral resolution of 1 \cminverse\ and a vertical resolution of 500 m up to model top at 50 km, and output optical depth, fluxes, heating rates, and absorption coefficients, all as a function of wavenumber and height. For simplicity in comparing to our analytical model below, RFM optical depth calculations assumed a zenith angle of 0, and fluxes and heating rates are computed using a two-stream approximation with a diffusivity factor of $D=1.5$ (rather than RFM's default 4-stream), as well as assuming constant $B(k,T)$ within atmospheric layers.  We omit the water vapor continuum in our base calculation, but consider its effects in Section \ref{sec_cont_co2}.  

% Shape of \chk
\subsection{The shape of \chk}
In real gas radiative transfer, clear-sky longwave radiative heating \ch\ is calculated by  integrating the spectrally-resolved heating 
\beqn
	\ch_k \ \equiv \ \frac{g}{\Cp} \ppp F_k \quad \text{(K/day/\cminverse)}
	\label{heat_k}
\eeqn
where $F_k$ is spectrally-resolved LW flux in $\Wmsq/\cminverse$. Thus,  any understanding of \ch\ must stem from an understanding of \chk. We begin by reviewing the qualitative picture of \chk\  found in, e.g., \cite{clough1992}.


Figure \ref{h2o_rfm_theory}c shows \chk\ from \htwo\ as computed by RFM on our idealized profile. There is a strong band of cooling for $k< 800\ \cminverse$, and a weaker band for $k > 1200\ \cminverse$. To understand these structures we consider   the optical depth $\tauk$, given by
\beqn
	\tauk(p) \ \equiv  \ \int_0^p \, \underbrace{\kappa(k,T,p')}_{\meter^2/\kg} \underbrace{\qv \frac{dp'}{g}}_{\kg/\meter^2}\ .
	\label{tauk}
\eeqn
Here \qv\ is water vapor specific humidity (kg/kg) and the mass absorption coefficient $\kappa$ ($\kg/\meter^2$) depends not only on $k$ but also on $T$ and $p$,  due to temperature-scaling and pressure broadening \citep[][]{pierrehumbert2010}. All other symbols have their usual meaning. Note that since $\kappa$ is an effective area per unit mass, and the rest of the integrand in \eqnref{tauk} is just the mass per unit area of absorber above level $p$ (i.e. the path length), $\tauk$ can be interpreted as the effective area of absorbers above level $p$ per unit (geometric) area.

 The spectral distribution of optical depth as output from RFM is shown in Fig. \ref{h2o_rfm_theory}b, which also plots the $\tauk=1$ levels for each $k$. The two diagonal bands in the \chk\ plot correspond to two diagonal $\tauk=1$ bands, which is where we expect emission at a given $k$ to maximize \citep[e.g.][we discuss the basis for this `$\tau=1$ law' below]{petty2006}. The shape of these $\tauk=1$ bands in the $k-p$ plane can themselves be understood in terms of reference absorption coefficients 
 \beqn
  \kapparef(k)\ \equiv  \ \kappa(k,\Tref,\pref)
  \eeqn
  where we take $(\Tref,\pref)=(300\ \Kelvin, 1\ \text{atm})$. These coefficients, also output from RFM, are shown in  Fig. \ref{h2o_rfm_theory}a. The two $\tauk=1$ bands  in Fig. \ref{h2o_rfm_theory}b correspond to the two absorption bands evident in the \kapparef\ plot: the pure rotation band ($k < 1000\ \cminverse$), and vibration-rotation band ($1000 < k< 1450 \ \cminverse$). By \eqref{tauk}, where \kapparef\ is relatively large then $\tauk=1$  occurs at relatively low pressures,  and vice-versa, so that plots of \kapparef\ and $\tauk=1$ levels must necessarily have the same shape, which then also manifest in the \chk\ field, all of which can be seen in the top row of Fig. \ref{h2o_rfm_theory}. 

% \tau=1
\subsection{The `$\tau=1$ law' and the cooling-to-space approximation} \label{sec_tau=1}
The $\tau=1$ law that we invoked above, however, strictly speaking applies only to the emission of OLR (or `cooling-to-space'), not radiative heating rates \emph{per se}; this is because  the latter includes not just cooling-to-space but also radiative exchange between atmospheric layers as well as the surface. However, it has been shown that these `exchange terms' are often negligible \citep{clough1992,rodgers1966}, so that radiative cooling may indeed be approximated solely by the cooling-to-space term. This leads to the  \emph{cooling-to-space approximation} \citep[e.g.][section 10.4]{petty2006}:  
\beqn
	\ppp F_k \ \approx \   \pi B(k,T) \partialder{\trans_k}{p} \  .
	\label{cts}
\eeqn
Here $\trans_k \ \equiv \ \exp(-\tauk)$ is the transmission function and $\pi B(k,T) \partialder{\trans_k}{p}$ is the cooling-to-space (or CTS) term in pressure coordinates, representing the differential contribution of an atmospheric layer to the \OLR\ at wavenumber $k$. In the CTS approximation then, spectrally-resolved radiative flux divergence is  simply Planck emission $\pi B(k,T)$ times a transmissivity gradient, the latter of which also serves as a weighting function since for optically thick wavenumbers its characteristic width and magnitude are constrained by $\int\ppp\trans_k \, dp = 1$. That the CTS term  maximizes at $\tau=1$ can be shown by  writing this transmissivity gradient as 
 \beqn
 		\partialder{\trans_k}{p} \  =  \ -\frac{\beta}{p} \tauk e^{-\tauk} 
		 \label{trans_grad} 
\eeqn
<<<<<<< Updated upstream
where
\beqn
		\beta \  \equiv  \ \partialder{\ln \tauk}{\ln p} \ . 
		\label{beta_def}
\eeqn
 The function $\tauk \exp(-\tauk)$ in \eqnref{trans_grad} has a relatively sharp peak at $\tauk=1$, and so long as $\beta/p$  has negligible vertical structure in the neighborhood of $\tau=1$, the transmissivity gradient and hence the CTS term will indeed peak at $\tau=1$ (as is approximately true for our gray radiation calculation; gray circle in Fig. \ref{ecmwf_vs_rfm}b). Note also that $\beta/p=\partialder{\ln\tauk}{p}$ is an inverse  `scale pressure' for optical depth, measuring how quickly \tauk\ increases by one $e$-folding, and as such  it simultaneously determines the width and magnitude of the transmissivity gradient. This quantity turns out to differ significantly between \htwo\ and \cotwo, as we will see in Section \ref{sec_co2}. 
 
 It should be noted here that the $\tau=1$ `law' for radiative cooling is not iron-clad, and can be violated in various ways. First, it will only hold if the \CTS\ term dominates, which is not a given; in PRE, for instance, the temperature profile adjusts itself until the exchange terms exactly cancel the \CTS\ term, as they must. Furthermore,  as emphasized by \cite{huang2014}, the \CTS\ term itself is only defined relative to a vertical coordinate $\xi$ (taken here to be pressure),  and  will only peak at $\tau=1$ if $\tau$ depends on $\xi$ exponentially or with a power law (as with $\xi=z$ or $p$), ensuring that $\partialder{\ln \tauk}{\xi}$ as in \eqnref{trans_grad} has little or no vertical structure. If one takes $\xi=\tau$, for instance, then one obtains $\partialder{\trans_k}{\tau} = e^{-\tau}$ which does not maximize at $\tau=1$. 
 

%
%To understand the relative magnitude of these terms, we first approximate all finite difference as derivatives, yielding schematically
%\beqn
%	\begin{split}
%		\CTS \ & \ \sim \  B \\
%		\SX	 \ & \ \sim \ \frac{d^2 B}{d \tau^2} \\
%		\AX	 \ & \ \sim \ \frac{d B}{d \tau} \\
%		\GX	 \ & \ \sim \ \frac{d B}{d \tau}  \ .
%	\end{split}
%	\label{cts_decomp_ders}
%\eeqn
=======
 The function $\tauk \exp(-\tauk)$ has a relatively sharp peak at $\tauk=1$, and so long as \partialder{\ln\tauk}{p}  has negligible vertical structure in the neighborhood of $\tau=1$, the transmissivity gradient and hence the CTS term will peak 
at $\tau=1$. Note also that \partialder{\ln\tauk}{p}\ is an inverse  `scale pressure' for optical depth, measuring how quickly \tauk\ increases by one $e$-folding, and that it simultaneously determines the width and magnitude of the transmissivity gradient. This quantity will play a key role in later sections.

Returning to heating rates, we now plot \chcts\ and \chkcts, the spectrally integrated and resolved heating rates calculated using  \eqnref{cts},  in Fig. \ref{coo_tau_kappa_h2o}a,b. Away from the surface, the CTS approximation appears to be quite accurate, as noted by previous studies \citep[e.g.][]{clough1992,rodgers1966}.  In addition, the \CTS\ term has a very simple mathematical structure,  and unlike the exact spectral cooling \eqref{heat_k} does not require an integration of the radiative transfer equations to find $F_k$. The \CTS\ term is thus much simpler to compute and manipulate than the total cooling, an advantage utilized by some GCM radiation parameterizations  \citep[e.g.][]{fels1980}.

The narrative so far ties together various pieces of radiative transfer physics in a satisfying way: spectrally-resolved heating rates \chk\ are dominated by the \CTS\ term \chkcts, which maximizes where $\tauk=1$, the height  of which is determined by \kapparef.  At the same time, however, this narrative does not yet provide a back-of-the-envelope estimate of \ch. Also, we still lack an understanding of why the \CTS\ approximation works so well, and under what conditions we might  expect it to break down.

To bring these questions into focus,  Fig. \ref{coo_tau_kappa_co2} shows plots analogous to that of Fig. \ref{coo_tau_kappa_h2o}, with the same idealized atmosphere except with \cotwo\ rather than \htwo\ as the radiatively active species, with a \cotwo\ concentration of 280 ppmv. The narrative of the previous paragraph largely holds, in that the bands of cooling correspond to $\tauk=1$ levels which themselves are determined by $\kapparef(k)$,  but some differences with the \htwo-only case emerge. Firstly, the mid-tropospheric spectrally integrated heating \ch\ for \cotwo\ is several  times smaller in magnitude than that for \htwo. Though this seems largely due to the width of their respective absorption bands, it is unclear how to make this  quantitative. Furthermore, the widths of these bands cannot be the whole story, as the spectrally-\emph{resolved} heating rates \chk\ for \cotwo\ are roughly half that of \htwo.  Finally, Fig. \ref{coo_tau_kappa_co2}a-c shows that the \CTS\ approximation holds less well for \cotwo\ than for \htwo. The reasons for this are unclear, as to date the CTS approximation seems only to have been justified by numerical experiment, rather than theoretical analysis. 

This review of the physics of radiative cooling has thus raised several questions:
\begin{enumerate}
	\item Why is $\ch\sim O(1\ \text{K/day})$? Can this be estimated in a back-of-the-envelope fashion? \label{Q_heating}
	\item Why does the CTS approximation work, and why does it work better for \htwo\ than \cotwo?	    \label{Q_cts}
	\item Why is $\ch_k$ twice as large for \htwo\ as for \cotwo, and why is \ch\ several times larger?		\label{Q_contrast}
\end{enumerate}

The goal of this paper is to answer these questions, by building and using simple radiation models which nonetheless emulate much more comprehensive radiative transfer schemes \citep[see][for further discussion of this approach]{jeevanjee2017a}. Our first task will be to understand the radiative cooling from a single spectral line (Section \ref{sec_single_line}); we do this in both a gray gas and real gas context , with the \CTS\ approximation playing a leading role in both. This single line analysis will yield answers to question \ref{Q_cts}. We then tackle the full spectrum of real gas radiation in Section \ref{sec_cts_theory}, where we combine the \CTS\ approximation with a parameterization of greenhouse gas spectroscopy inspired by that of \cite{wilson2012}. These simplifications will allow us to spectrally integrate the right-hand side of \eqref{cts} and build simple models for $\ch_k$ and \ch, allowing us to answer questions \ref{Q_heating} and \ref{Q_contrast}.

%%================%
%% Single-line analysis  %
%%================%
%\section{Radiative cooling from a single line} \label{sec_single_line}
%We begin by analyzing radiative cooling from a single line. At first glance, intuition for this quantity  seems difficult to come by, even in a gray gas context, as heating rates are flux divergences (Eqn. \ref{heat}) so one must solve the radiative transfer equations first to obtain both upwards and downwards fluxes, and then differentiate to get the heating rate. This makes radiative cooling, in principle, a non-local quantity. The great simplification of the \CTS\ approximation is that \eqnref{cts} is \emph{local} in the vertical, and thus much simpler to compute and, perhaps, to understand. We will develop a formalism for the \CTS\ approximation and give a brief theoretical justification for it in Section \ref{sec_cts_gray},  deferring full details to Appendix \ref{appendix_cts}. We will then apply the \CTS\ approximation to real gases in Section \ref{sec_cts_lbl}, using it to answer question \ref{Q_cts} and part of \ref{Q_contrast} and also lay the groundwork for the spectral analysis of Section \ref{sec_cts_theory}.
%
%%========%
%% cts_gray   %
%%=========%
% 
%\subsection{The \CTS\ approximation} \label{sec_cts_gray}
%We use a gray gas model \citep{pierrehumbert2010} to establish our formalism and analyze the \CTS\ approximation, as this model contains the essential physics. For simplicity and clarity we will work with optical depth   as our vertical coordinate, taking care to point out where other coordinates may give differing behavior. 
%
%Consider a gray source function $B$ in  \Wmsq,  optical depth $\tau$ increasing downwards towards a surface value of \taus, and net upward flux $F$ satisfying the gray radiative transfer equations. As detailed in Appendix \ref{appendix_cts}, the flux divergence (in $\tau$ coordinates) may be decomposed as 
%	\beqn
%		\ddtau{F}(\tau) \ = \ \CTS \ + \ \SX + \ \AX \ + \  \GX \  . 
%		\label{cts_decomp}
%	\eeqn
%These terms are depicted schematically in Fig. \ref{cts_decomp_cartoon}, and can be interpreted as follows. CTS is the `cooling-to-space'  term 
%	\beqn
%		\CTS \ \equiv \ - B(\tau) e^{-\tau} 
%	\label{cts_def}
%	\eeqn
%which represents the energy emitted by a layer to outer space (Fig. \ref{cts_decomp_cartoon}a). 
%% (the \CTS\ approximation is just the claim that this term dominates \eqref{cts_decomp}).
%% i.e. that
%%\beqn
%%	\pptau F \ \approx \ - B(\tau)e^{-\tau} \ .
%%	\label{cts_approx}
%%\eeqn
%%Tthe other terms in \eqref{cts_decomp} are as follows.  
%The \SX\ term is the `symmetric exchange' term 
%	\beqn 
%		\SX \ \equiv \   \ \left\{ \begin{array}{cr} \int_0^\tau [B(\tau + x) - 2B(\tau) + B(\tau-x)]e^{-x}\,dx   & \tau < \taus/2 \\
%																															 &  \\	
%												\int_{0}^{\taus-\tau} [B(\tau + x) - 2B(\tau) + B(\tau-x)]e^{-x}\,dx   & \tau > \taus/2 
%							\end{array}   \right.  
%			\label{sx1}
%	\eeqn
% and represents exchange between level $\tau$ and layers both above and below with equal optical thickness (Fig. \ref{cts_decomp_cartoon}b). Exchange with the colder layer above at $\tau-x$ yields a first-order finite difference $B(\tau-x) - B(\tau)$, and vice-versa for the warmer layer below, yielding the second-order finite difference in \eqnref{sx1}.  \citep[This term  gives rise to the `diffusive' approximation to radiative cooling found in textbooks, e.g.][]{goody1989}. Meanwhile, \AX\ is the `anti-symmetric exchange' term
%	\beqn
%		\AX  \ \equiv \  \ \left\{ \begin{array}{cr} \int_{\tau}^{\taus-\tau}[B(\tau + x)-B(\tau)]e^{-x}	dx 	& \tau < \taus/2 \\
%																																					& \\
%													 \int_{\taus - \tau}^{\tau}[B(\tau - x)-B(\tau)]e^{-x} 	dx 	& \tau > \taus/2 
%							\end{array}  \right. \quad 
%				\label{ax1}
%	\eeqn
% which represents exchange between level $\tau$ and that part of the atmosphere not included in \SX, which will lie entirely at either greater or smaller $\tau$ values (Fig. \ref{cts_decomp_cartoon}c). Thus \AX\  contains only a first-order finite difference.
%Finally,  \GX\ is the `ground-exchange' term
%\beqn
%	\GX\  \equiv  \  [B(\Ts) - B(\tau)]\exp[-(\taus - \tau)] \ .
%	\label{gx1}
%\eeqn
%representing exchange between level $\tau$ and the surface (Fig. \ref{cts_decomp_cartoon}c).
%
%(As an important aside, note that CTS term \eqref{cts_def} has weighting function $e^{-\tau}$, which on its own does \emph{not} maximize at $\tau=1$; as emphasized by \cite{huang2014}, the  `$\tau=1$ law' arises from considering a flux divergence \emph{relative to a vertical coordinate} $\xi$ (such as $z$ or $p$) on which $\tau$ depends exponentially or with a power law, ensuring that $\partialder{\ln \tauk}{\xi}$ as in \eqnref{trans_grad} has little or no vertical structure.)
%
%To understand the relative magnitude of these terms, we first approximate all finite difference as derivatives, yielding schematically
%\beqn
%	\begin{split}
%		\CTS \ & \ \sim \  B \\
%		\SX	 \ & \ \sim \ \frac{d^2 B}{d \tau^2} \\
%		\AX	 \ & \ \sim \ \frac{d B}{d \tau} \\
%		\GX	 \ & \ \sim \ \frac{d B}{d \tau}  \ .
%	\end{split}
%	\label{cts_decomp_ders}
%\eeqn
>>>>>>> Stashed changes
%Thus, the \CTS\ term is distinguished by the fact that it represents one-way exchange to space, and is thus proportional to $B$ rather than a derivative. To better quantify this, we need a mathematical form for $B(\tau)$, which can be obtained by combining our  constant lapse rate atmosphere (with $T\sim p^{\Rd\Gamma/g}$) with the commonly used power-law form for $\tau(p)$
% \beqn
% 	\tau = \taus(p/\ps)^\beta 
%	\label{taup}
%\eeqn 
%and the fact that $B\sim T^4$ for a gray gas. Putting this together, we find
%\beqn
%	B(\tau) = B(\taus)(\tau/\taus)^\gamma
%	\label{Btau1}
%\eeqn
% where
%  \beqn
% 	\gamma \ = \ \frac{4R_d\Gamma}{g\beta} \ .
%	\label{gamma_gray}
%\eeqn
%Thus $\gamma$ determines how rapidly thermal emission varies with optical depth, and from \eqnref{cts_decomp_ders} we see that the exchange terms will be enhanced/suppressed by one or more factors of $\gamma$ relative to the \CTS\ term. The \CTS\ approximation will thus be justified if 
%\begin{align}
%	\hspace{6cm}  \gamma  \ll 1 \  \hspace{2cm}  \text{(criterion for CTS approximation) }
%	\label{cts_criterion}
%\end{align}
%(A more careful analysis, which reaches the same conclusion, is given in Appendix \ref{appendix_cts}). Plugging in $\Gamma = 7\ \Kelvin/\km$ and  typical values of $\beta=2$ for a well-mixed gray gas with pressure broadening (P10) and $\beta=4$ for a condensable gas  \citep[e.g.][]{frierson2006,held1982}, we find
%\beqa
%\gamma_{\text{well-mixed}}    & = & 0.4 \\
%\gamma_{\text{condensable}} & = & 0.2  \ .
%\eeqa
%Thus the \CTS\ approximation should hold reasonably well  for our gray gas analog of \htwo,  and less so for our \cotwo\ analog, with the difference being due to the differences in the power law exponent $\beta =  d \ln\tau/d \ln p$. In the next section we will compute $\beta$ and $\gamma$ for real \htwo\ and \cotwo, and use these values to  justify the \CTS\ approximation (or not) as well as understand these gases' single-line radiative cooling profiles.
%
%%=========%
%% cts_lbl        %
%%==========%
%
%\subsection{CTS approximation: line-by-line analysis } \label{sec_cts_lbl}
%The gray analysis performed above may alternately be viewed as an analysis of spectrally resolved radiative cooling at a single wavenumber $k$, except that for the source function $B$ we must use the spectral  Planck function $B(k,T)$, and more importantly we must critically evaluate the appropriateness of the parametrization \eqnref{taup} and our values of $\beta$ against a line-by-line benchmark, which to our knowledge has not been done. 
%
<<<<<<< Updated upstream
=======
%For this purpose we use the Reference Forward Model \citep[RFM,][]{dudhia2017}, a flexible, line-by-line, longwave radiative transfer model. We used HITRAN spectroscopic data for \htwo\ from 0--1500 \cminverse\ and \cotwo\ from 500--850 \cminverse, and used an idealized atmospheric profile with $\Ts=300\ \Kelvin$,  a constant lapse rate of $\Gamma= 7\ \Kelvin/\km$ up to to an isothermal stratosphere at $\Tstrat=200$ K, and \RH=0.75 for \htwo\ calculations and a \cotwo\ concentration of 280 ppmv, except where specified. We ran RFM at a spectral resolution of 1 \cminverse\ and output optical depth, heating rates, and absorption coefficients, all as a function of wavenumber and pressure. For simplicity in comparing to our analytical model, optical depth calculations assumed a zenith angle of 0, and heating rates were computed using a two-stream approximation (rather than RFM's default 4-stream) as well as assuming constant $B(k,T)$ within atmospheric layers. Also for simplicity we omitted the water vapor continuum, though the effects of this are explored in Section \ref{sec_real_atm}.  RFM's $\chi$ factor was used to suppress far-wing absorption of \cotwo. 
>>>>>>> Stashed changes
%
%
%To apply the ideas of the preceding section, we first note that from \eqnref{Btau1} we have
%\beqa
%	\gamma  & = &\ \frac{d \ln B}{d\ln \tau} \n \\
%			   & =   &\left(\frac{d \ln B}{d \ln T}\right) \left(\frac{d \ln T}{d \ln p}\right) \left(\frac{d \ln \tau}{d \ln p}\right)^{-1} \n \\
%			&  \equiv & \alpha \frac{\Rd \Gamma}{g} \frac{1}{\beta} \label{gamma_real}
%\eeqa
%where $\alpha \equiv \frac{d \ln B}{d \ln T}$.  Near 600 \cminverse, a wavenumber relevant for both \htwo\ and \cotwo, $\alpha \approx 3$ (not shown), so we take this as a characteristic value. Note that this is not too far from the gray gas value of $\alpha = (d \ln \sigma T^4/d \ln T) =  4$ (compare Eqns. \eqnref{gamma_gray} and \eqref{gamma_real}). 
%
%As for $\beta$, we plot $\frac{d \ln \tau}{d \ln p}$ diagnosed directly from RFM output in Fig. \ref{beta}, and calculate values and uncertainties for $\beta$ as the mean and standard deviation of $\frac{d \ln \tau}{d \ln p}$ restricted to the $\tau=1$ lines. To understand the results, we proceed hierarchically. Fig. \ref{beta}a shows $\frac{d \ln \tau}{d \ln p}$ for a simple case of \cotwo\ only, assuming an isothermal atmosphere at 200 K.  The most common values are those just below 2, with  $\beta=1.7$, reflecting the usual pressure-broadening enhancement of absorption coefficients away from line centers \citep{coakley2014}. Values of $\frac{d \ln \tau}{d \ln p}$ significantly less than 2 or even 1 are evident for some $k$ values, however, reflecting the differing behavior near line center. If we consider \cotwo\ only but with our simplified troposphere, we find that temperature-scaling increases $\beta$ to roughly 2.7 (Fig. \ref{beta}c), so we set 
%\begin{subequations}
%	\beqn
%		\beta_{\cotwo} = 2.7.
%		\label{beta_co2}
%	\eeqn
%
%For \htwo\ we find that without temperature scaling but with the Clausius-Clapeyron (CC) scaling of \rhov, $\beta = 5$, not far from the $\beta=4$ often assumed in the literature. (Note that stratospheric $\frac{d \ln \tau}{d \ln p} < 2 $ values for \htwo\ are much less common for \htwo\ rather than \cotwo, reflecting the much greater line spacing for \htwo.) Including temperature scaling, however, brings $\beta$ up to 5.7, and we thus set 
%	\beqn
%		\beta_{\htwo} = 5.7.
%		\label{beta_h2o}
%	\eeqn
%	\label{beta_vals}
%\end{subequations}
%
%With $\beta$ values in place and with $\Rd \Gamma/g = 0.2$, we find 
%\begin{subequations}
%	\begin{align}
%		\gamma_{\cotwo} & \approx 0.2 \\
%		\gamma_{\htwo} & \approx 0.1  \ .
%	\end{align}
%	\label{gamma_vals}
%\end{subequations}
%These values of $\gamma$ are about half of the simple gray values we deduced in \eqnref{gamma_gray}.  Thus the gray theory is not terribly accurate but gets us in the ballpark, for roughly the right reasons, and thus reinforces the conclusion from the the previous section that the CTS approximation should hold well for water vapor, and somewhat less so for \cotwo. 
%
<<<<<<< Updated upstream
%To check this in detail, Fig. \ref{cooling_profiles} shows profiles of each term in \eqnref{cts_decomp} (multiplied by $\frac{g}{\Cp}\der{\tauk}{p}$ to get a heating rate), for both \htwo\ and \cotwo, and for optical depths corresponding to emission levels of 300, 550, and  800 hPa. These profiles were produced by feeding \tauk\ profiles output from RFM into an offline code which numerically evaluates  Eqns. \eqnref{sx1} and \eqnref{ax1} (the CTS and GX terms are straightforwardly evaluated from \eqnref{cts_def} and \eqnref{gx1}). The CTS approximation indeed works remarkably well for \htwo\, and less so for \cotwo\ across the spectrum, due to the differing values of $\gamma$ in \eqnref{gamma_vals}, which themselves result from the differing values of $\beta$ in \eqnref{beta_vals}. Also, the \cotwo\ $\ch_k$  profiles are much broader and have roughly half the amplitude compared to the \htwo\ profiles, as we also saw in Figs. \ref{h2o_rfm_theory}b and \ref{coo_tau_kappa_co2}b; this is again a consequence of the differing $\beta$ values, since the inverse optical depth scale pressure discussed below \eqnref{trans_grad} is here given by
=======
%To check this in detail, Fig. \ref{cooling_profiles} shows profiles of each term in \eqnref{cts_decomp} (multiplied by $\frac{g}{\Cp}\der{\tauk}{p}$ to get a heating rate), for both \htwo\ and \cotwo, and for optical depths corresponding to emission levels of 300, 550, and  800 hPa. These profiles were produced by feeding \tauk\ profiles output from RFM into an offline code which numerically evaluates  Eqns. \eqnref{sx1} and \eqnref{ax1} (the CTS and GX terms are straightforwardly evaluated from \eqnref{cts_def} and \eqnref{gx1}). The CTS approximation indeed works remarkably well for \htwo\, and less so for \cotwo\ across the spectrum, due to the differing values of $\gamma$ in \eqnref{gamma_vals}, which themselves result from the differing values of $\beta$ in \eqnref{beta_vals}. Also, the \cotwo\ $\ch_k$  profiles are much broader and have roughly half the amplitude compared to the \htwo\ profiles, as we also saw in Figs. \ref{coo_tau_kappa_h2o}b and \ref{coo_tau_kappa_co2}b; this is again a consequence of the differing $\beta$ values, since the inverse optical depth scale pressure discussed below \eqnref{trans_grad} is here given by
>>>>>>> Stashed changes
%\beqn
%	\der{\ln \tauk}{p} \  \approx \ \frac{\beta }{p} \ .
%	\label{scale_beta}
%\eeqn
%Thus, question \ref{Q_cts} and part of question \ref{Q_contrast} can be answered as follows: The CTS approximations works better for  \htwo\ than for  \cotwo\ because $\beta_{\cotwo} < \beta_{\htwo}$, which also implies that the \cotwo\ weighting function and hence cooling will be weaker and more broadly distributed in the vertical. These different $\beta$ values are ultimately attributable to \cotwo\ being well-mixed versus \htwo\ being condensable and governed by C-C. 
%
%With this understanding of radiative cooling from a single line and the CTS approximation in place, we turn to the full spectrum of greenhouse gas radiation in the next section.

%The analytical formalism we develop in the next section will allow us to make this statement precise, and also to estimate $\beta$ analytically. 

%=========%
<<<<<<< Updated upstream
% h2o_theory  %
%==========%
\section{A simple model for  \htwo\ cooling} \label{sec_h2o_theory}

% h2o_spectral
\subsection{Spectrally-resolved cooling \chk} \label{sec_h2o_spectral}
To build a simple model for radiative cooling from \htwo,  we must first simplify its spectroscopy, i.e. the functional form of its mass absorption coefficients $\kapparef(k)$ (Fig. \ref{h2o_rfm_theory}a). We do this by following \cite{wilson2012} and first coarse-graining $\ln \kapparef(k)$ over spectral intervals of $10 \ \cminverse$. We then delineate the two relevant \htwo\ spectral bands as
\beqa
	\begin{split}
	    	\text{\htwo\ rotation band (\textbf{rot}): } & & \krot \equiv 150 & <\  k\  < \ 1000\ \cminverse \\
    		\text{\htwo\ vibration-rotation band (\textbf{vr}): } & &  1000 & < \  k\ <\  1450\ \cminverse \equiv \kvr   \ .
	\end{split}
	\label{h2o_bands}
\eeqa
 We then apply a linear fit to $\ln \kapparef(k)$ within each band  to obtain a piecewise exponential approximation for $\kapparef(k)$:
=======
% cts_theory  %
%==========%
\section{An analytical model for spectrally-integrated cooling} \label{sec_cts_theory}
 The previous sections have analyzed the CTS approximation for spectrally-resolved heating $\ch_k$, thus addressing question \ref{Q_cts} posed in the introduction, and also part of question \ref{Q_contrast}. In this section we build on those results and spectrally integrate them to obtain a simple model for spectrally-integrated heating $\ch$, with an eye towards fully answering questions \ref{Q_heating} and \ref{Q_contrast}.
 
 To build a simple model for radiative cooling from real gases, we must first simplify their spectroscopy, i.e. the functional form of their mass absorption coefficients $\kapparef(k)$ (Fig. \ref{kappa}, thin dotted light gray lines). We do this by following \cite{wilson2012} and first coarse-graining $\ln \kapparef(k)$ for each gas over spectral intervals of $10 \ \cminverse$ (Fig. \ref{kappa},  thick dashed dark gray lines). We then define two spectral bands for each gas as follows:
\begin{subequations}
	\beqa
	    	\text{\htwo\ rotation band (\textbf{rot}): } & \krot \equiv 150 & <\  k\  < \ 1000\ \cminverse \\
    		\text{\htwo\ vibration-rotation band (\textbf{vr}): } & 1000 & < \  k\ <\  \kvr\equiv 1450\ \cminverse \\
    		\text{\cotwo\ $P$ band: } & 500 &<\   k\ <\  \kQ\equiv 667.5\ \cminverse \\
    		\text{\cotwo\ $R$ band: } & \kQ &<\   k\ < \  850\ \cminverse 
	\eeqa
	\label{band_defs}
\end{subequations}
 \citep[here \kQ\ denotes the spectral location of the \cotwo\ $Q$ branch, which lies between the $P$ and $R$ branches but has a much smaller spectral width, e.g.][]{coakley2014}. We than apply a linear fit to $\ln \kapparef(k)$ within each band  to obtain approximations for $\kapparef(k)$:
 \begin{subequations}
>>>>>>> Stashed changes
 \beqa
 	\kappa_{\htwo}(k) & \equiv & \left\{ \begin{array}{cr} 
													\kapparot \exp\left(-\frac{k-\krot}{\lrot}\right) \quad \quad \text{for $k$ in \textbf{rot}}  \\
												    \kappavr \exp\left(-\frac{\kvr-k}{\lvr}\right)   \quad \quad \text{for $k$ in \textbf{vr} .}
												      \end{array} \right.          
\label{kappa_h2o}
 \eeqa
This approximation is shown in Fig. \ref{h2o_rfm_theory}d, and by construction emulates the gross behavior of $\kapparef(k)$. The parameters $\ln\kapparot$ and $\ln \kappavr$ are obtained as the maxima of the corresponding straight line fits, and $\lrot$ and $\lvr$ as the corresponding slopes. These parameter values are tabulated in Table \ref{band_params}. Note that the $l$ parameters have units of \cminverse, and describe how fast $\kapparef$ exponentially declines with $k$.
 
With the simplified absorption spectra \eqref{kappa_h2o} in hand we can now construct simplified expressions for \htwo\  optical depth \tauk. This is done using the analytical expression for water vapor path from  \cite{koll2018} and  including an approximate pressure-broadening factor $p/\pref$ (but still neglecting temperature scaling), which yields
\begin{align}
	 \tauk & = \kappa_{\htwo}(k)\frac{p}{\pref}\WVP_0 \exp\left(-\frac{L}{\Rv T}\right) \ .
	\label{tauk_h2o} 
\end{align}
Here $\WVP_0$ has units of water vapor path and is given by
\beqn
	\WVP_0\equiv D \frac{\Tav \RH p_v^\infty}{\Gamma L}\ ,  
	\label{WVP0}
\eeqn
where $\Tav = (\Ts+\Tstrat)/2$ is an average tropospheric temperature and $D$ is our diffusivity factor of 1.5. 
%\comment{Nadir: Ensure consistency of \Tav\ vs. $T$ in all numeric calculations} !!!
 The optical depth distribution \eqnref{tauk_h2o} is shown in Fig. \ref{h2o_rfm_theory}e. While it does not capture the fine structure in the RFM output (by construction), it does seem to capture the gross optical depth distributions. As a further test of \eqnref{tauk_h2o}, we may plug it into the CTS approximation \eqnref{cts} to obtain a spectrally simplified \chk, shown in Fig. \ref{h2o_rfm_theory}f. Our simplified \chk\ captures the broad characteristics of the \chk\ produced by RFM, though its peak values are somewhat larger. This is not surprising however, as our simple model ignores all but the largest scale variations in $\kappa(k)$. The finer scale variations we neglect serve to `smear out'  cooling in  the $k-p$ plane, such that the cooling around a given $(k,p)$ occurs over a larger $(k,p)$ range in RFM than in our simple model, and must thus have a smaller magnitude in RFM. We will see below that we nonetheless get reasonable values for the spectrally integrated cooling \ch. Taken as a whole, then, Figure \ref{h2o_rfm_theory} shows that our simple model seems to capture the gross behavior of our spectrally-resolved RFM calculations, using as input parameters only the $\kappa$ and $l$ parameters listed in Table \ref{band_params}.

% h2o_integrated
\subsection{Spectrally-integrated cooling \ch} \label{sec_h2o_integrated}
 We now construct a simple model for spectrally integrated cooling \ch, with an eye towards capturing and understanding the typical value of $\ch\approx -2$ K/day. This will require two further derived quantities. The first is an expression for $\beta$ for \htwo, which we obtain by substituting our expression \eqnref{tauk_h2o} for $\tauk$  into the definition \eqnref{beta_def} of $\beta$:
\beqn
	\beta_{\htwo} \ = \ 1+ \frac{L}{\Rv T}\frac{\Gamma \Rd}{g} \ .
	\label{beta_h2o}
\eeqn
Typical tropospheric values for $\Gamma=7$ K/km are $\beta_{\htwo}=5.5 \pm 1$.

The second (and more important) derived quantity is the wavenumber profile $\konej(p)$, which is the wavenumber in band $j$ (where $j$ denotes either the \rot\ or \vr\  bands from \eqnref{h2o_bands}) which has $\tauk=1$ (and hence cools-to-space) at a given height.  As discussed in Section \ref{sec_tau=1}, these $\tau=1$ contours give the locus of cooling in the $k-p$ plane. We obtain analytical expressions for \konej\  by substituting our simplified $\kappa(k)$ fit  Eqn.  \eqnref{kappa_h2o} into the $\tauk$ formula Eqn. \eqnref{tauk_h2o}, setting $\tauk=1$, and solving for $k$ in each band, yielding
 \beqa
	\begin{split}
	 	\konerot & = & \krot + \lrot\left[\ln(\kapparot\WVP_0) \    + \ \ln(p/\pref) \  - \ \frac{L}{\Rv T} \right]  \\ 
		\konevr  & = &  \kvr \ - \ \lvr\left[\ln(\kappavr\WVP_0) \ \  + \ \ln(p/\pref) \  - \ \frac{L}{\Rv T} \right]   
	\end{split} \quad .
	\label{k1_theory}
\eeqa
These $\tau=1$ contours  are overlaid over the simple $\tauk$ distribution in Fig. \ref{h2o_rfm_theory}e, and seem to  capture the overall shape and $x$ and $y$ intercepts of the noisier $\tau=1$ contours diagnosed from RFM output (Fig. \ref{h2o_rfm_theory}b). Note that the exponentials in $\kappa_{\htwo}(k)$ and $\exp(-L/\Rv T)$ cancel out in \eqnref{k1_theory}, leaving \konej\ with a strong dependence on $T$ and only a logarithmic dependence on other variables, which will be of significance later.

With \emph{mise en place}  we can now pursue our goal of an analytical expression for the spectrally integrated heating in band $j$, denoted $\ch_j$. We begin by integrating \eqnref{heat_k} over band $j$, assuming the limits of the integral are implicitly given by  the appropriate wavenumber range from \eqnref{h2o_bands}, and invoking the CTS approximation \eqnref{cts} as well as Eqn.  \eqnref{trans_grad}. This yields
 \beqn
\ch_j \ =   \ -  \frac{g}{\Cp}\frac{\beta}{p}\int dk\,  \pi B(k,T)\, \tauk \exp(-\tauk) \ .
	\label{heat_cts1}
\eeqn
 To evaluate this integral we note that $\int_0^\infty d\tauk\,\tauk\exp(-\tauk)=1$ and that $\tauk\exp(-\tauk)$ peaks at $\tauk=1$, with corresponding $k$ value $\konej$. This suggests that we might approximate $\tauk\exp(-\tauk)$ by the Dirac delta function $\delta(\tauk -1)$. This can be plugged into \eqnref{heat_cts1} once we convert this to a delta function in $k$-coordinates, using the appropriate chain rule \citep[e.g.][]{gasiorowicz2003} as well as Eqns. \eqref{tauk_h2o} and \eqnref{kappa_h2o}:
    \begin{align*}
               \delta(\tau_k- 1) \ = \ \left|\partialder{\tau_k}{k}(\konej)\right|\inverse\!\!\delta(k-\konej) \ = \ \lj\,  \delta(k-\konej)  \ .
      \end{align*}
Note the appearance of the \lj\ parameter here. Plugging this equation into \eqnref{heat_cts1} and performing the now trivial spectral integration yields finally our desired expression for band-wise integrated radiative cooling,
\beqn
		\ch_j  \ \approx \ - \frac{g}{\Cp}\pi B(\konej,T)\frac{\beta}{p} \lj \ .
	\label{heat_cts3}
\eeqn
Note that $B(\konej,T)$ gives the Planck emission as a function of height \emph{only,} since $\konej$ gives the wavenumber which cools-to-space at a given height (cf. \eqnref{k1_theory}). Section \ref{sec_2k} below will give an explicit evaluation of this formula.

% Interpretation
\subsection{Interpretation}
Equation \eqnref{heat_cts3} is a central result of this paper. How should we interpret it? We know from \eqnref{cts} that the CTS term is just Planck emission times the transmissivity gradient $\partialder{\trans_k}{p}$. Using Eqn. \eqnref{trans_grad} we evaluate this gradient at  $\tauk=1$, yielding
\beqn
	\left. \partialder{\trans_k}{p} \right |_{\tauk=1}  \ =  \ -\ \frac{1}{e}\frac{\beta}{p} \ .
	\label{trans_grad_tau1}
\eeqn
Apart from the factor of $e$, this expression appears in \eqnref{heat_cts3}. Also, we will argue momentarily  that the transmissivity gradient can equivalently be interpreted as  the  `emissivity-to-space' gradient; when combined with \eqnref{trans_grad_tau1}, this allows us to write \eqnref{heat_cts3} as 
	\beqn
		\ch_j \ = \ -\frac{g}{\Cp} \underbrace{\pi B(\konej,T)}_{\substack{ \text{Planck } \\ \text{emission}\\(\Wmsq/\cminverse) } }
					   \underbrace{\left(\frac{1}{e}\frac{\beta}{p}\right)}_{\substack{ \text{emissivity} \\ \text{gradient}  \\ (1/\Pa) } } 
					   \underbrace{(\lj e)}_{\substack{  \text{spectral} \\ \text{width} \\ (\cminverse) } }    \; .
		\label{heat_cts4}
	\eeqn
 To interpret the transmissivity gradient factor in this way, note that for an atmospheric layer of thickness $\Delta p$ we can interpret $-\Delta p\der{\trans_k}{p}  = \Delta p (d \tauk/dp)\trans_k$ as its `emissivity to space', since $\Delta p(d \tauk/dp)$ gives the absolute  emissivity of the layer and $\trans_k$ is the fraction of emitted radiation that escapes to space. Thus $-\der{\trans_k}{p}$ can be interpreted as the emissivity-to-space gradient (in pressure coordinates). 

As for the factor $\lj e$ in \eqnref{heat_cts4}, it can be interpreted as the characteristic  width of the spectral region at any given height that is cooling to space. For the \htwo\ \rot\ band this gives a width of 165 \cminverse,  in rough eyeball agreement with the width of the active cooling regions in Fig. \ref{h2o_rfm_theory}c,f.  Thus, one interpretation of  \eqnref{heat_cts4} is that radiative heating at a given height  may be interpreted as just spectral Planck emission, times a spectral width,  times an appropriate measure of emissivity,  all evaluated at \konej\ where $\tau_{\konej}=1$. This yields a flux divergence (which in pressure coordinates has units $\Wmsq/\Pa$), which can be multiplied by $g/\Cp$ to get a heating rate in K/day.
 
 \section{Validation and sensitivities}
 Now let us test  whether $\ch_j$ can emulate the heating profiles generated by RFM. Figure \ref{H1d_rfm_cts_theory}a shows  $\sum_j \ch_j$ (as calculated via \eqnref{heat_cts3}) for \htwo, along with \ch\  output directly from RFM, as well as the spectrally integrated CTS approximation \eqnref{cts} using \tauk\ output from RFM. The CTS approximation matches \ch\ quite well throughout most of the troposphere, as is well-known \citep{clough1992,stephens1984,rodgers1966}, though the CTS approximation significantly underestimates cooling near the surface. Since our simple model \eqnref{heat_cts3} is based on the CTS approximation it also underestimates cooling near the surface, as well as  overestimates cooling in the upper troposphere. Also, our simple model collapses all cooling onto the $\tauk=1$ contours shown in Fig. \ref{h2o_rfm_theory}, and the resulting $\ch_{\rot}$ profile thus cuts off abruptly near 200 hPa, rather than smoothly transitioning to lower values near the tropopause (and then back to larger values in the upper stratosphere, as the RFM profiles do). There is a smaller but similar discontinuity near 300 hPa where the $\ch_{\vr}$ contribution drops out. Despite these errors and approximations, however, the simple model nevertheless exhibits the fairly constant $\sim 2$ K/day value produced by more comprehensive calculations, as well as the usual `kink' in the upper troposphere (discussed further in Section \ref{sec_kink} below). Note that no parameters have been tuned to obtain this agreement. 
 
 To further test Eqn. \eqnref{heat_cts3}, as well as study the sensitivity of \ch\ to the humidity and temperature variations seen across the globe, we perturb our atmospheric column. In one perturbation calculation we change \RH\ from 0.75 to 0.3, and in another we change the lapse rate $\Gamma$ from  7 to 5 K/km.  We  run RFM and also evaluate \eqnref{heat_cts3} on these two perturbed atmospheres, with the results shown in Figs. \ref{H1d_rh_gamma_Ts}a,b  (we cut-off the RFM profiles at the same height that the simple model profiles terminate, for clarity). Both the RFM and simple model profiles show a marked insensitivity to \RH. Both models also show a reduction in cooling in the middle and lower troposphere of roughly $30\%$ with the reduction in $\Gamma$ (in the upper troposphere this signal becomes convolved with that from the differing upper-tropospheric temperatures). These relatively small sensitivities, which are consistent with  the relative uniformity of clear-sky \ch\  across the globe (Fig. \ref{ecmwf_vs_rfm}), can be understood using Eqns. \eqnref{heat_cts3} as well as \eqnref{k1_theory}. These show that changes in \RH\ only affect \konej, decreasing $\WVP_0$ by roughly a factor of two and hence changing the \konej\ by only $\lj\ln 2\approx 35\ \cminverse$, not enough to significantly change the Planck emission $B(\konej,T)$. Changing $\Gamma$, on the other hand,  yields similar changes in  $\konej$ but also has the additional effect of decreasing $\beta$ by roughly $30\%$  (cf. Eqn. \eqnref{beta_h2o}), leading to a similar reduction in \ch. Physically, lowering the vertical temperature gradient  lowers  water vapor emissivity gradients because of Clausius-Clapeyron, which lowers the flux divergence per unit pressure and hence the heating rate. 

As a further  test we consider atmospheric columns with varying $\Ts=(270,280,290,300)$ K, leaving $\RH$ and $\Gamma$ unchanged from the base case. We again run RFM and also evaluate \eqnref{heat_cts3} on these atmospheres, with the results shown in Figs. \ref{H1d_rh_gamma_Ts}c.d. The simple model captures the variations on \ch\ with \Ts, and in particular captures the weakening of the kink in the upper troposphere with decreasing \Ts. This gives some confidence that the simple model exhibits a kink for the right reasons, allowing use to study the kink using the simple model in Section \ref{sec_kink} below.

%==============%
% sec_applications %
%==============%
\section{Applications} \label{applications}

% 2k
\subsection{A back-of-the-envelope estimate of \ch} \label{sec_2k}
 We now use \eqnref{heat_cts3} to make a back-of-the-envelope estimate of \ch, as follows.   Taking $(T,p)=(260\ \Kelvin,500\ \hPa)$, which corresponds to $\konerot=500\ \cminverse$ and $\konevr=1350\ \cminverse$, we find
\beqa
	 \pi B(\konerot,260\ \Kelvin) & \approx & 0.3 \ \Wmsq/\cminverse \n \\ 
	 \pi B(\konevr,260\ \Kelvin)  & \approx  & 0.05 \ \Wmsq/\cminverse \n \ .
\eeqa 
Thus \vr\ Planck emission is roughly 1/6 of that for \rot, which explains why the cooling in the \vr\ band in Fig. \ref{h2o_rfm_theory}c,f is much smaller than that in the \rot\ band. We thus neglect $\ch_\vr$ for this estimate, and  also take $\beta_\htwo \approx 5$. We then have
\beqanonum
	\ch & \approx & \ch_\rot \\
		  & \approx &- \frac{g}{\Cp}\pi B(\konerot,T)\, \lrot \frac{\beta_{\htwo}}{p} \\
		  & \approx &- \left(\frac{10\ \frac{\meter}{\second^2}}{10^3 \frac{\joule}{\kg\ \Kelvin}}\right) \left(0.3 \ \Wmsq/\cminverse\right)(60 \ \cminverse)\left(\frac{5}{5\times 10^4 \ \Pa}\right) \\
		  & \approx & - \left(\frac{10\ \frac{\meter}{\second^2}}{10^3 \frac{\joule}{\kg\ \Kelvin}}\right)(20\ \Wmsq)\left(\frac{1}{10^4\ \Pa}\right)  \\
		  & = & -\ 2 \times 10^{-5}\ \Kelvin/\second \\
		  & \approx & -\ 2\ \text{K/day} \ .
\eeqanonum
 Thus the formula \eqnref{heat_cts3} indeed gives us a way to quickly estimate \ch, using only fundamental constants,  a typical value of the Planck function, and the single RFM-derived parameter \lrot, which characterizes  water vapor spectroscopy.

% Kink
\subsection{The upper-tropospheric kink} \label{sec_kink}
  Both RFM and  our simple model \eqnref{heat_cts3} reproduce the well-known relatively sharp kink in \ch\ in the upper troposphere, which has been attributed to a decline in water vapor emissivity\footnote{Note that this notion is not very well-defined for a real gas, as the emissivity or transmissivity is a function of wavenumber.} driven by a drop off in \qv, and which has served as the basis for the Fixed Anvil Temperature hypothesis \citep[FAT; e.g.][]{hartmann2002, hartmann2001}. To the degree that \eqnref{heat_cts3} captures this for the right reasons, however, the kink is \emph{not} a result of \qv\ declining to levels too low to efficiently cool to space. Rather, Eqn. \eqnref{heat_cts3} tells us that vertical variations in \ch\ stem from the $\beta/p$ factor, which increases by several fold as one ascends through the troposphere, and the Planck function $B(\konerot,T)$, which decreases by several fold (Fig. \ref{planck_k1rot}; we continue to neglect the \vr\ band). These two effects largely compensate throughout most of the troposphere, with both quantities changing  slowly in the lower troposphere and more rapidly higher up. This compensation ends at  300 hPa, however, when the declining Planck function wins out, yielding the kink. According to our simple model, then, it is the decline of the Planck function (due to both declining temperatures and wavenumbers) which is responsible for the upper tropospheric decline in cooling, rather than any Clausius-Clapeyron driven decline in \qv\ and emissivity.
  
% Ts-invariance
\subsection{\Ts-invariance of $\ppt F$} \label{sec_Ts_invariance}
It was recently pointed out in \cite{jeevanjee2018} (hereafter JR18) that radiative flux divergences, when computed in temperature coordinates, exhibit a certain invariance with respect to \Ts: the $\ppt F(T)$ profiles for various \Ts\ all collapse onto a common curve, where the curve simply extends to larger $T$ as \Ts\ increases. This `\Ts-invariance' of $\ppt F$ has implications for column-integrated cooling and precipitation change (JR18), so it seems worth verifying that \Ts-invariance also holds for the idealized radiative cooling considered here, and that our simple model also captures it.

Figure \ref{pptf_tinv} shows $\ppt F(T)$ (\Wmsq/K) for our atmospheric columns with variable \Ts, as computed by RFM and also via 
\beqn
		\ppt F  \ \approx \ - \pi B(\konej,T)\frac{1}{T}\left(\frac{L}{\Rv T}+\frac{g}{\Rd\Gamma}\right) \lj \ 
	\label{pptf_simple}
\eeqn
which is just \eqnref{heat_cts3} multiplied by $(C_p/g)(dp/dT)$. Though the RFM profiles at various \Ts\ all show the pronounced lower-tropospheric cooling which is neglected by the CTS approximation and hence absent in our simple model, both calculations nevertheless exhibit a high degree of  \Ts-invariance. Together with the results of \cite{cronin2017}, this provides some sense that the \Ts-invariance of $\ppt F$ is robust. Figure \ref{pptf_tinv}  also shows that our simple model captures the relevant physics, which is simply that \eqnref{pptf_simple} is essentially  a function of $T$ alone. The only $p$-dependence in \eqnref{pptf_simple} are the pressure broadening factors in \eqnref{k1_theory} (which were neglected in the argument of JR18), but these  are only logarithmic in $p$ and hence can't compete with the  $T$-dependence stemming from the Planck function and Clausius-Clapeyron.
 
 Another point to make about these profiles is that the flux divergence in temperature coordinates is not roughly constant in height, but is rather monotonic, with no kink in the upper troposphere. Thus the kink in $\ch\sim \ppp F$  results from choosing pressure as the vertical coordinate, with the pressure derivative yielding a $1/p$ factor in \eqnref{heat_cts3} which yields the kink, as discussed above.
 
 
% OLR
\subsection{OLR} \label{sec_olrk}
The formalism developed here can also be applied to estimate the spectrally resolved outgoing longwave radiation $\OLRk \equiv F_k(p=0)$, where we estimate this as simply the Planck function evaluated at an appropriate emission temperature. For tropospheric emission we estimate the emission temperature by setting $\tauk=1$ in  \eqnref{tauk_h2o} and solving for $T$; analytical expressions for the resulting emission temperatures $T_1(k)$ in terms of the Lambert $W$ function are given in Appendix \ref{appendix_T1}. For some $k$, however,  $T_1(k)$ will be undefined because $\tauk<1$ even at the surface; this is the water vapor `window' region $\krot(\Ts) < k < \kvr(\Ts)$, for which we set \OLRk\ equal to surface Planck emission.  Mathematically, our estimate for  \OLRk\ is then 
\beqn
	\OLRk \ =  \ \left\{ \begin{array}{cl} \pi B(k,T_1(k)) & \mbox{where $T_1(k)$ is defined} \\
														\pi B(k,\Ts) & \mbox{where $\krot(\Ts) < k < \kvr(\Ts)$ \quad (window region) \ .} 
								\end{array}						
					   \right .
	\label{eqn_olrk}
\eeqn
Figure \ref{olr} shows this estimate of \OLRk, along with \OLRk\ as output directly from RFM. The above estimate, while crude, quantitatively captures the gross spectral shape of RFM's \OLRk. Furthermore, it gives us some insight into this shape, as follows. Both the RFM and simple model \OLRk\ curves peak at the beginning of the window region, $\krot(\Ts) \approx 750\ \cminverse$, and the simple  \OLRk\ estimate in particular has a cusp. This is because beyond $\krot(\Ts)$,   the emission temperature is no longer increasing with $k$ but rather becomes constant at \Ts,  allowing the explicit $k$-dependence of $B(k,T)$ to take over and cause an immediate and sharp decline in \OLRk. Thus the peak in \OLRk\ is not pegged to the $k$-maximum of $B(k,T)$ for some $T$, as might be suspected (in fact, $B(k,300\ \Kelvin)$ maximizes at a lower wavenumber of  $k\approx 590\ \cminverse$), but is rather pegged to the onset of the water vapor window. Adding the \htwo\ continuum to our RFM calculation does not change this conclusion (Fig. \ref{olr}, dotted line).


% CO2
\subsection{\cotwo\ } \label{sec_co2}
As another application of the formalism developed so far, we pursue an idealized calculation of $\tauk$ and \chk\ for \cotwo, analogous to those shown in Fig. \ref{h2o_rfm_theory} for \htwo. This application illustrates the generality of the formalism, and also contrasts the behavior of the two gases.

For this case we run RFM just as described in Section \ref{sec_rfm_calcs}, except our $k$ range is now 500-850 \cminverse, we use a preindustrial \cotwo\ concentration of 280 ppmv, and RFM's $\chi$ factor was used to suppress far-wing absorption of \cotwo. In analogy to  \eqnref{h2o_bands}, we define bands for \cotwo:
\beqa
		\begin{split}
	    		\text{\cotwo\ $P$ band: } && 500 &<\   k\ <\  \kQ\equiv 667.5\ \cminverse \\
    			\text{\cotwo\ $R$ band: } & & \kQ &<\   k\ < \  850\ \cminverse  \ 
		\end{split}
		\label{co2_bands}
\eeqa
 \citep[here \kQ\ denotes the spectral location of the main \cotwo\ $Q$ branch, which lies between its associated $P$ and $R$ branches but has a much smaller spectral width, e.g.][]{coakley2014}. This band structure can be seen in $\kapparef(k)$ as output from RFM, shown in Figure \ref{co2_rfm_theory}a. We then coarse-grain $\ln \kapparef(k)$ over spectral intervals of $10 \ \cminverse$ and apply a linear fit to $\ln \kapparef(k)$ within each band. These linear fits give very similar slopes and maxima, so we combine the two fits into the single expression 
  \beqn
 	\kappa_{\cotwo}(k)  \equiv   \kappaQ \exp\left(-\frac{|k-\kQ|}{\lQ}\right) \quad \quad \text{for $k$ in $P$ or $R$,}   			
	\label{kappa_co2}  
\eeqn
 %\eqnref{kappa_co2}, yielding a   approximation for $\kapparef(k)$:
 where $\ln \kappaQ$ and  \lQ\ are averages of the maxima and slopes from the $P$ and $R$ bands. These parameter values are also tabulated in Table \ref{band_params}, and the corresponding fits are shown in  Fig. \ref{co2_rfm_theory}c. Note that \cotwo's $\lQ$ parameter, which governs its characteristic spectral width, is roughly 1/5 of \lrot,  which says that at any given level \cotwo\ emits in a much narrower spectral range than \htwo. 

With Eqn.  \eqref{kappa_co2} in hand we now construct simplified expressions for  \cotwo\ optical depth. Evaluating  \eqnref{tauk} with pressure broadening but no temperature scaling and constant \cotwo\ concentration $q$ (rather than variable \qv) yields
\beqn
	\quad \tauk  = \kappa_{\cotwo}(k)\frac{qp^2}{2g\pref}   \ .
	\label{tauk_co2}
\eeqn
 A comparison between these optical depth distributions and those output from RFM are given in Fig.  \ref{co2_rfm_theory}b,d. As for \htwo, our approximations do not capture the fine structure in the RFM output but do capture the gross optical depth distribution. Note also that Eqns. \eqnref{tauk_co2} and \eqnref{beta_def} imply 
 \beqn
 	\beta_{\cotwo} \ = \ 2 \ ,
	\n
\eeqn
a value 2-3 times smaller than that for \htwo\ (cf Eqn. \eqnref{beta_h2o}).

We can now plug \eqnref{tauk_co2} into \eqnref{cts} to obtain a spectrally simplified \chk, shown in Fig. \ref{co2_rfm_theory}f. As for \htwo, our simplified \chk\ captures the broad characteristics of the \chk\ produced by RFM, including the relatively large values of \chk\ in the stratosphere. From Eqns. \eqnref{cts} and \eqnref{trans_grad} with $\beta_{\cotwo}=2$  we see that this stratospheric enhancement of \chk\ must stem from the low values of $p$ in \eqnref{trans_grad}. Another view of this is that any wavenumber whose $\tauk=1$ level is at low $p$ must have a high $\kapparef$ and hence high optical depth and transmissivity gradients.

It is also worth noting that the tropospheric \chk\ values for \cotwo\ are roughly 2-3 smaller than those for the \htwo\ \rot\ band. 
This can be understood from Eqns. \eqnref{cts}-\eqnref{trans_grad}, which tell us that this is due to the factor of 2-3 difference in $\beta$ between the two gases. In other words, because \cotwo\ optical depth does not increase with pressure as fast as \htwo, its cooling at any given wavenumber is more spread out in the vertical and is thus smaller. This decrease in \chk\ is compounded by \cotwo's smaller value of $l$, which as pointed out above means that the effective width with which it cools any given layer is $\lQ e \approx 35 \ \cminverse$, roughly 1/5 the characteristic width of \htwo. This small width can be confirmed by inspection of Figs. \ref{co2_rfm_theory}c,f, and implies that spectrally-integrated radiative cooling \ch\ from \cotwo\ will only be a fraction of that from \htwo. We could push further and construct idealized \cotwo\ \ch\ profiles  (as we did for \htwo), but due to their relative insignificance in the troposphere we stop here. 

%============%
% sec_cont_co2  %
%============%
	
\subsection{Effects of \htwo\ continuum and \cotwo\ overlap} \label{sec_cont_co2}
Our calculations so far have neglected the water vapor continuum as well as overlap effects between \htwo\ and \cotwo, both of which are known to affect radiative cooling and OLR. While we do not pursue simple models of these effects here, we should investigate the errors we induce by neglecting them, and explain why our 1D RFM calculation which neglects these effects nonetheless resembles the ECMWF profile in Fig. \ref{ecmwf_vs_rfm}b, which includes them.

We first consider the \htwo\ continuum. Figure \ref{cont_co2_effects}a shows \chk\ from an RFM calculation identical to the base case considered above, but with the continuum turned on \citep[RFM contains a hard-coded version of the MT\_CKD continuum,][]{mlawer2012}. The spectrally integrated \ch\ profile from this case is shown in Fig. \ref{cont_co2_effects}c. The continuum increases \ch\ throughout the troposphere, especially at lower levels. When we then add on the effects of \cotwo\ overlap (Fig. \ref{cont_co2_effects}b,c), however, we find that much of this increase is cancelled due to the presence of strong \cotwo\ lines. From Fig. \ref{cont_co2_effects}c we see that  the only real contrast between our base case and the more realistic case with both continuum and \cotwo\ contributions is at temperatures of 285 or above (thinking here in temperature coordinates). Such temperatures  may not necessarily make a strong contribution to the globally averaged \ch\ profile shown in Fig. \ref{ecmwf_vs_rfm}, which is averaged on pressure levels and thus conflates different temperatures.

%=========%
% Summary  %
%=========%
\section{Summary and discussion} \label{sec_summary}
By approximating greenhouse gas spectroscopy with piecewise linear functions (Eqns. \eqnref{kappa_h2o} and \eqnref{kappa_co2}) as well as employing analytically expressions for \htwo\ and \cotwo\ optical depth (Eqns. \eqnref{tauk_h2o} and \eqnref{tauk_co2}), we built simple models of spectrally resolved and integrated radiative cooling as well as OLR which roughly emulate the gross behavior of these quantities as simulated by a state-of-the-art, line-by-line radiative transfer scheme (Figs. \ref{h2o_rfm_theory}--\ref{olr}). The simplicity of our models allowed us to answer various questions about the magnitude of radiative cooling (i.e. why 2 K/day), the sensitivity of cooling profiles to perturbations, the nature of the upper-tropospheric kink in \ch\ and the  spectral peak in OLR, \Ts-invariance of $\ppt F$, and the contrast in cooling between \htwo\ and \cotwo.

Our results also give perspective on the oft-used `Newtonian cooling' approximation to radiative cooling \cite[e.g.][]{fueglistaler2009}. In this picture, net radiative cooling results from temperature perturbations away from a steady-state radiative equilibrium  temperature profile. At the same time, the CTS approximation \eqnref{cts}  suggests that LW cooling is \emph{not} determined relative to any radiative equilibrium profile, but is just given by the Planck function  times an emissivity gradient, which never vanishes. The resolution to this is that in the stratosphere, where Newtonian cooling approximations are often employed, LW cooling is balanced by SW heating, primarily from ozone. Thus the radiative equilibrium of the stratosphere is not a pure radiative equilibrium in the LW (in which the CTS approximation would break down due to cancellation by exchange terms), but rather a balance between the LW and SW. Thus the base state of a Newtonian cooling approximation indeed has significant LW cooling, as implied by \eqnref{cts} and also recognized in earlier literature \citep[e.g.][]{dickinson1973}.

Our formalism can also be combined with other theories. In section \ref{sec_Ts_invariance} we made contact with the theory of JR18 but there are additional possibilities, such as the recently published semi-analytical formalism of \cite{koll2018}. As a quick example, we make a rough estimate of their parameter $\Tinf\approx 350 \ \Kelvin$, which is the temperature at which the no-continuum water vapor window closes in a saturated, moist-adiabatic atmosphere. To do that within our formalism, we take Eqn. \eqnref{tauk_h2o} and first set $k=1000\ \cminverse$ (center of the window), corresponding via Eqn. \eqnref{kappa_h2o} to a global minimum absorption coefficient $\kappamin\approx2\times 10^{-4}\ \meter^2/\kg$. We also set $p=\pref$ so that we are at the surface, and plug $\RH=1$ and $\Gamma=2.5$ K/km \citep[estimated from the $\Ts=350$ K adiabat in Fig. S1  of ][]{goldblatt2013} into Eqn. \eqnref{WVP0} (keeping $\Tav=250$ K). We then solve the resulting equation for temperature, yielding the  estimate
\beqn
	\Tinf \ = \ \frac{L}{\Rv \ln(\kappamin\WVP_0)} \ \approx \ 360 \ \Kelvin,
	\n
\eeqn
in the neighborhood  of  350 K as obtained by \cite{koll2018} with a comprehensive line-by-line calculation.

While such applications are encouraging, the simple model as developed here is still highly idealized and could be further developed in several different directions. Further developments could include incorporating the effects of the \htwo\ continuum and \cotwo\ overlap, as discussed in  Section \ref{sec_cont_co2}. One could also similarly simplify the spectroscopy of other important greenhouse gases like methane and ozone, and thus include their effects in this simple model. With sufficient elaboration, such efforts could lead towards transparent, extremely cheap, reduced-accuracy interactive radiation schemes. This could be done at various levels within the framework presented here, but even the use of the smoothed spectroscopy  alone (without the simplified optical depth expressions which assume constant \RH\ and $\Gamma$) could reduce the cost of existing radiation calculations. For applications where high-accuracy radiation is not required, such as short-term weather forecasting or idealized modeling \citep[e.g. idealized aquaplanets such as that of][]{frierson2006}, the reduction in accuracy may be acceptable.

Another extension of this work would be towards less idealized atmospheric profiles. As emphasized by, e.g., \cite{stevens2017},  vertical gradients in \RH\ can have dramatic implications for radiative cooling profiles and their associated  circulations. Similar effects have been studied in the context of radiative instabilities and self-aggregation of convection \citep{beucler2018,beucler2016,emanuel2014}. Extending this work to include non-uniform \RH\ might allow for greater understanding and  ease of modeling of phenomena such as these. 

% Discuss CTS approximation in revision, once have a better sense of sensitivity to vertical grid.

\appendix
\section*{Appendix}
\section{Emission temperatures for \htwo} \label{appendix_T1}
To estimate \OLRk\ we need $\Tone(k)$, the effective emission temperature as a function of wavenumber. We estimate this by  setting $\tauk=1$ in  \eqnref{tauk_h2o} and solving for $T$, where we first substitute in
\beqn
	p \ =\ \ps\left(\frac{T}{\Ts}\right)^{g/\Rd\Gamma}  
	\n
\eeqn
for $p$. This yields a transcendental equation for $T$ which can nonetheless be solved using the Lambert $W$-function, which satisfies $W(xe^x) = x$, i.e. it inverts the function $xe^x$. After some algebraic manipulation of Eqn. \eqnref{tauk_h2o} to put it into the form $xe^x$, we obtain
\beqn
	\Tone(k)  \ = \ 	\frac{\Tstar}{W(\frac{\Tstar}{\Ts}\tau_0^{\Rd\Gamma/g})}
	\n
\eeqn
where 
\beqnonum
	\Tstar 	  \ \equiv \ \frac{\Rd \Gamma L}{g \Rv} \ , \hspace{1cm}  \tau_0(k)  \ \equiv \  \WVP_0\kappa(k)\frac{\ps}{\pref}  \ .
\eeqnonum
	
%========%
% Figures    %
%========%

\begin{table}[h]
	\begin{center}
		\begin{tabular}{c | | c | c} 
							  		  			  & \htwo  & \cotwo \\ \hline \hline
			$\beta$							  & $\beta_{\htwo} \ = \ 5.5$  & $\beta_{\cotwo} \ = 2$  \\  \hline
			$k$ (\cminverse)			  & \begin{tabular}{@{}c@{}}\krot = 150 \\ \kvr = 1450  \end{tabular} & \kQ = 667.5  \\  \hline
			$\kappa\ (\meter^2/\kg)$ & \begin{tabular}{@{}c@{}}\kapparot = 250\\ \kappavr = 10 \end{tabular} & \kappaQ = 175 \\ \hline  
			$l$ (\cminverse)			  & \begin{tabular}{@{}c@{}}\lrot = 61 \\ \lvr = 42  \end{tabular} & \lQ = 13  \\  \hline
		\end{tabular}
		\caption{Parameters used in our analytical models. The $k$ parameters are chosen for \htwo\ by inspection of Fig. \ref{h2o_rfm_theory}a, and for \cotwo\ to be proximate to the central 667.66 \cminverse\ $Q$ branch line.The $\kappa$ and $l$ parameters, on the other hand, result from fits to RFM output. See text for details.
		\label{band_params}
		}
	\end{center}
\end{table}

% Figure ecmwf_vs_rfm
\begin{figure}[h]
	\begin{center}
			\includegraphics[scale=0.7]{../plots/ecmwf_vs_rfm}
			\caption{\textbf{(Left)}  Zonal mean heating rates \ch\ from ECMWF reanalysis for June-July-August 2001. \ch\ has a characteristic value of -2 K/day which is fairly uniform in height and longitude. 
						\textbf{(Right)} Vertical heating rate profiles as calculated by globally averaging the ECMWF \ch\ (solid black line) and by performing a 1D calculation on our idealized atmosphere using both RFM (dashed black line)  and  a gray model (solid gray line). The $\tau=1$ level for the gray profile is denoted with a filled gray circle. The 1D RFM calculation emulates the ECMWF global mean, but the 1D gray calculation does not.
			\label{ecmwf_vs_rfm}
			}
	 \end{center}
\end{figure}

%Figure h2o_rfm_theory
\begin{figure}[h]
	\begin{center}
			\includegraphics[scale=0.42]{../plots/h2o_rfm_theory}
		\caption{Comparison between RFM output and simple model, as follows:
					 \textbf{(a,d)} \htwo\ absorption spectrum \kapparef, at (\Tref, \pref) = (300 K, 1 atm) from RFM and our linear fit, respectively
					 \textbf{(b,e)} Logarithm of diffusion parameter $D$ times optical depth $\tauk$, from RFM and Eqn. \eqnref{tauk_h2o}, respectively, along with $\tauk=1$ contours.
					 \textbf{(c,f)}  Spectrally resolved heating $\ch_k$, from RFM and theory (Eqns. \eqnref{cts} and \eqnref{tauk_h2o}). 
					 All plots show averages over 10 \cminverse\ bins. These panels show that the spectrally resolved cooling \chk\ can be understood as emission from $\tauk=1$ levels, where the height of these levels is determined by $\kapparef(k)$, and that  our simple model captures this physics.
		\label{h2o_rfm_theory}
		}
	\end{center}
\end{figure}


%Figure H1d_rfm_cts_theory
\begin{figure}[h!]
	\begin{center}
			\includegraphics[scale=0.7]{../plots/H1d_rfm_cts_theory.pdf}
		\caption{ Heating rate profiles calculated directly from RMF (black solid line), using RFM optical depths in conjunction with the CTS approximation \eqnref{cts} (solid gray line), and as  $\sum_j\ch_j$ using our simple model \eqnref{heat_cts3}.  Despite some errors, the simple model captures  the overall magnitude and upper tropospheric kink of the RFM \ch\ profile. See text for discussion.
	  \label{H1d_rfm_cts_theory}
		}
	\end{center}
\end{figure}

%Figure H1d_rh_gamma_Ts
\begin{figure}[h]
	\begin{center}
			\includegraphics[scale=0.6]{../plots/H1d_rh_gamma_Ts}
		\caption{\textbf{(a)} Profiles of \ch\ as output from RFM applied to columns with $\RH=0.3$ (blue line) and $\Gamma=5$ K/km (red line). 
					\textbf{(b)}\ As in (a), but computed from \eqnref{heat_cts3}. 
					\textbf{(c)}\ As in (a), but for atmospheres with varying \Ts.
					\textbf{(d)}\ As in (c), but computed from \eqnref{heat_cts3}. 
					All RFM profiles are cut-off where the corresponding simple model profile cuts off, for clarity of comparison. The simple model appears to capture the sensitivities of our RFM calculation to variations in \RH, $\Gamma$, and \Ts.
		\label{H1d_rh_gamma_Ts}
		}
	\end{center}
\end{figure}

%Figure Planck_k1rot
\begin{figure}[h]
	\begin{center}
			\includegraphics[scale=0.7]{../plots/planck_k1rot}
		\caption{\textbf{(Left)} The Planck density $\pi B(k,T)$, with $\konerot$ overlaid. \textbf{(Right)} The function $B^*(\konerot,T)$, equal to $B(\konerot,T)$ normalized by its surface value, along with $\beta^*/p^*$, equal to the emissivity gradient from Eqn. \eqnref{heat_cts4} normalized by its surface value.  Note logarithmic $x$-axis. The change in both factors roughly compensate throughout most of the troposphere up until $\approx 300$ hPa or so, where the decline in $B^*(\konej,T)$ is more severe. This yields the well-known kink in \ch. 
		\label{planck_k1rot}
		}
	\end{center}
\end{figure}

%Figure pptf_tinv
\begin{figure}[h]
	\begin{center}
			\includegraphics[scale=0.7]{../plots/pptf_tinv}
		\caption{Test of the \Ts-invariance of flux divergences $\ppt F(T)$, for both RFM (left) and our simple model (right) when expressed in temperature coordinates [Eqn. \eqnref{pptf_simple}]. \Ts-invariance holds for our 1D RFM calculations away from the surface, and this is captured by the simple model.
		\label{pptf_tinv}
		}
	\end{center}
\end{figure}

%Figure olr
\begin{figure}[h]
	\begin{center}
			\includegraphics[scale=0.7]{../plots/olr}
		\caption{Spectrally resolved outgoing longwave radiation \OLRk, as computed from RFM (black line) and Eqn. \eqnref{eqn_olrk} (red line). Equation \eqnref{eqn_olrk} captures the shape of RFM's \OLRk, and also shows that the peak in \OLRk\ is due to the onset of the water vapor window. This conclusion is unchanged by comparing to RFM's \OLRk\ computed with the \htwo\ continuum (dotted line).
		\label{olr}
		}
	\end{center}
\end{figure}

%Figure co2_rfm_theory
\begin{figure}[h]
	\begin{center}
			\includegraphics[scale=0.42]{../plots/co2_rfm_theory}
		\caption{As for Fig. \ref{h2o_rfm_theory}, except for the 550-850 \cminverse\ \cotwo\ band.
		\label{co2_rfm_theory}
		}
	\end{center}
\end{figure}



%Figure cont_co2_effects
\begin{figure}[h]
	\begin{center}
			\includegraphics[scale=0.45]{../plots/cont_co2_effects}
		\caption{Spectrally resolved radiative cooling \chk\ as output from RFM with \textbf{(a)} the \htwo\ continuum included and \textbf{(b)} \htwo\ continuum plus \cotwo\ absorption. Panel \textbf{(c)} shows the spectrally integrated cooling rates \ch\ for these cases plus the base case. Continuum emission enhances \ch, particularly in the lower troposphere, but this effect is largely canceled out by \cotwo\ absorption. 
		\label{cont_co2_effects}
		}
	\end{center}
\end{figure}

\pagebreak

\bibliographystyle{apa}
\bibliography{/Users/nadir/Dropbox/resources/bibtex_mendeley/library}


\end{document}

