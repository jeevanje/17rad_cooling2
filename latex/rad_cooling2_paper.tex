%% Draft settings
\documentclass[10pt]{article}
\usepackage{amsmath}
\usepackage{amssymb}
\usepackage{graphicx}
\usepackage{subfigure}
\usepackage{color}
 \usepackage{lineno}
\usepackage{simplemargins}
\usepackage{natbib}

% \linenumbers*[1]
 \usepackage[T1]{fontenc} % For citing barki{\dj}ija
\setkeys{Gin}{draft=false}


% Margins
\setleftmargin{1in}
\setrightmargin{1in}
\setbottommargin{1in}
\settopmargin{1in}
 
% My Commands
%\input{tensor_book_shrtcts.tex}
\newcommand{\comment}[1]{\textcolor{blue}{[{#1}]}}


%%Nadir's Shortcuts
\newcommand{\beqn}{\begin{equation}}
\newcommand{\eeqn}{\end{equation}}
\newcommand{\beqa}{\begin{eqnarray}}
\newcommand{\eeqa}{\end{eqnarray}}
\newcommand{\beqanonum}{\begin{eqnarray*}}
\newcommand{\eeqanonum}{\end{eqnarray*}}
\newcommand{\beqnonum}{\begin{equation*}}
\newcommand{\eeqnonum}{\end{equation*}}
\newcommand{\jump}{\vspace{0.5cm}}
\newcommand{\bbf}{\begin{bf}}
\newcommand{\ebf}{\end{bf}}
\newcommand{\eqnref}[1]{(\ref{#1})}
\newcommand{\defn}[1]{\begin{bf}\emph{#1}\end{bf}}
\newcommand{\reals}{\ensuremath{\mathbb{R}}}
\newcommand{\complex}{\ensuremath{\mathbb{C}}}
\newcommand{\integers}{\ensuremath{\mathbb{Z}}}
\newcommand{\half}{\ensuremath{\frac{1}{2}}}
\newcommand{\n}{\nonumber}
\newcommand{\inverse}{^{-1}}

%calculus shorthand
\newcommand{\timeder}{\frac{d}{dt}}
\newcommand{\partialder}[1]{\frac{\partial}{\partial #1}}
\newcommand{\partialderf}[2]{\ensuremath{\frac{\partial #1}{\partial #2}}}
\newcommand{\der}[2]{\ensuremath{\frac{d #1}{d #2}}}
\newcommand{\dx}{\ensuremath{\frac{d}{dx}}}
\newcommand{\ddx}{\ensuremath{\frac{d}{dx}}}
\newcommand{\ddtau}[1]{\ensuremath{\frac{d #1}{d\tau}}}
\newcommand{\kvec}{\ensuremath{\vec{k}}}
\newcommand{\uvec}{\ensuremath{\mathbf{u}}}
\newcommand{\zhat}{\ensuremath{\mathbf{\hat{z}}}}
\newcommand{\khat}{\ensuremath{\mathbf{\hat{k}}}}
\newcommand{\unitvect}[1]{\ensuremath{\mathbf{\hat{#1}}}}
\newcommand{\ppx}{\ensuremath{\partial_x}}
\newcommand{\ppy}{\ensuremath{\partial_y}}
\newcommand{\ppz}{\ensuremath{\partial_z}}
\newcommand{\ppt}{\ensuremath{\partial_T}}
\newcommand{\ppp}{\ensuremath{\partial_p}}
\newcommand{\pptau}{\ensuremath{\partial_\tau}}


% radiation shorthand
\newcommand{\cotwo}{\ensuremath{\mathrm{CO_2}}}
\newcommand{\htwo}{\ensuremath{\mathrm{H_2O}}}
\newcommand{\QLW}{\ensuremath{Q_\mathrm{LW}}}
\newcommand{\QSW}{\ensuremath{Q_\mathrm{SW}}}
\newcommand{\Qnet}{\ensuremath{Q_\mathrm{net}}}
\newcommand{\Qcts}{\ensuremath{Q_\mathrm{cts}}}
\newcommand{\Qgex}{\ensuremath{Q_\mathrm{gex}}}
\newcommand{\FLW}{\ensuremath{F}}
\newcommand{\FSW}{\ensuremath{F^\mathrm{SW}}}
\newcommand{\Fnet}{\ensuremath{F^\mathrm{net}}}
\newcommand{\olr}{\ensuremath{\mathrm{OLR}}}
\newcommand{\OLR}{\ensuremath{\mathrm{OLR}}}
\newcommand{\trans}{\ensuremath{\mathcal{T}}}
\newcommand{\cool}{\ensuremath{\mathcal{C}}}
\newcommand{\cminverse}{\ensuremath{\mathrm{cm^{-1}}}}
\newcommand{\pierre}{P10}
\newcommand{\pem}{\ensuremath{p_1}}
\newcommand{\tauk}{\ensuremath{\tau_k}}
\newcommand{\taus}{\ensuremath{\tau_s}}
\newcommand{\Bs}{\ensuremath{B_s}}
\newcommand{\SX}{\ensuremath{\mathrm{SX}}}
\newcommand{\AX}{\ensuremath{\mathrm{AX}}}
\newcommand{\CTS}{\ensuremath{\mathrm{CTS}}}

% meteorology shorthand
\newcommand{\qv}{\ensuremath{q}}
\newcommand{\rhov}{\ensuremath{\rho_\mathrm{v}}}
\newcommand{\Hv}{\ensuremath{H_\mathrm{v}}}
\newcommand{\Rv}{\ensuremath{R_\mathrm{v}}}
\newcommand{\qa}{\ensuremath{q_a}}
\newcommand{\qvstar}{\ensuremath{q^*}}
\newcommand{\Ta}{\ensuremath{T_a}}
\newcommand{\Tav}{\ensuremath{T_\mathrm{av}}}
\newcommand{\Ts}{\ensuremath{T_\mathrm{s}}}
\newcommand{\ps}{\ensuremath{p_s}}
\newcommand{\RH}{\ensuremath{\mathrm{RH}}}
\newcommand{\WVP}{\ensuremath{\mathrm{WVP}}}
\newcommand{\ztop}{\ensuremath{z_\mathrm{top}}}
\newcommand{\ztp}{\ensuremath{z_\mathrm{tp}}}
\newcommand{\zlcl}{\ensuremath{z_\mathrm{LCL}}}
\newcommand{\Tlcl}{\ensuremath{T_\mathrm{LCL}}}
\newcommand{\Ttp}{\ensuremath{T_\mathrm{tp}}}
\newcommand{\ptp}{\ensuremath{p_\mathrm{tp}}}
\newcommand{\lapseav}{\ensuremath{\Gamma_\mathrm{av}}}
\newcommand{\gammatr}{\ensuremath{\Gamma_\mathrm{tr}}}
\newcommand{\gammast}{\ensuremath{\Gamma_\mathrm{st}}}
\newcommand{\Kinverse}{\ensuremath{\mathrm{K^{-1}}}}
\newcommand{\Htauk}{\ensuremath{H_{\tau_k}}}

%Variables
\newcommand{\figurepath}{../figures/}


\begin{document}

%% ------------------------------------------------------------------------ %%
%
%  TITLE
%
%% ------------------------------------------------------------------------ %%


\title{On the cooling-to-space from water vapor and carbon dioxide}

%% ------------------------------------------------------------------------ %%
%
%  AUTHORS AND AFFILIATIONS
%
%% ------------------------------------------------------------------------ %%


 \author{Nadir Jeevanjee and Stephan Fueglistaler}

\maketitle

\begin{abstract}
The contribution of water vapor emission to  top-of-atmosphere radiance (i.e. water vapor's `cooling-to-space', or CTS) emanates largely from the troposphere, whereas the CTS from carbon dioxide emanates largely from the stratosphere. Furthermore, these CTS terms  roughly approximate the flux divergence in a given layer, giving rise to the `CTS approximation' for radiative cooling. Here, we use  a simple grey model to investigate the fundamental reasons for these effects. We find that the differences in emission level between \htwo\ and \cotwo\ are due not spectroscopy but rather to differences in their respective scale heights, and that the CTS approximation relies on this as well as the magnitude of the tropospheric and stratospheric lapse rates. We also find that the CTS approximation holds for \htwo\ across its spectrum, but fails for some parts of the \cotwo\ specturm.  Analytical expressions are developed to provide intuition for these results, and their relevance  is assessed using comprehensive  line-by-line radiative transfer  calculations.


%\vspace{0.5cm}
%
%
\end{abstract}


%% ------------------------------------------------------------------------ %%
%
%  TEXT
%
%% ------------------------------------------------------------------------ %%


\section {Introduction}
Spectrally and vertically resolved radiative cooling (plotted in K/day/\cminverse\ in $k-p$ coords in Fig. \ref{lbl_H}) for a tropical column show that the \htwo\ rotational band and \cotwo\  vibrational band behave markedly differently: the \cotwo\ band cools primarily in the stratosphere, whereas the \htwo\ band cools primarily in the troposphere. Furthermore, if we plot the contribution of each $k-p$ layer to the TOA radiance at that $k$ (Fig. \ref{lbl_cts}), we find that this approximates the radiative cooling very well; this is the `cooling-to-space' or CTS approximation \citep{rodgers1966, clough1992,thomas2002}.

Two questions immediately spring to mind here: why do the \htwo\ and \cotwo\ bands behave so differently? And why does the CTS approximation work so well? We seek to answer these questions by emulating the LBL results with a single-line grey radiative transfer model, whose behavior can be understood analytically. 

%================%
% Model construction  %
%================%
\section{The CTS approximation}
	%\subsection{The verticall
To understand the differences between \htwo\ and \cotwo\ emission, as well as to analyze the validity of the CTS approximation, we construct a simple grey radiation model to emulate spectrally-resolved radiative transfer for both gases. We consider a single band optical depth $\tau(p)$ (increasing downward) given by
\beqn
	\tau(p) = \taus(p/\ps)^\beta
	\label{tau_p}
\eeqn
where $p$ is pressure and subscript `s' denotes surface values. We will consider two cases: $\beta = 1$, corresponding to a well-mixed greenhouse gas such as  \cotwo, and $\beta = 4$, yielding a scale height \comment{define/explain} which is 1/4 that of the atmosphere, or roughly 2 km, comparable to that found for \htwo\   \citep{romps2014}. This difference in $\beta$ will play a critical role, as we will see. We assume a surface temperature $\Ts=300$ K, and constant tropospheric and stratospheric lapse rates of $\gammatr = 6.5$ K/km and $\gammast = -2$ K/km on either side of a tropopause with temperature $\Ttp=200$ K and corresponding  pressure $\ptp = 114$ hPa \comment{confirm}.  Thus  
\beqn
	T(p) = \left\{ \begin{array}{cc} \Ts(p/\ps)^{R_d \gammatr/g} & p > \ptp \\
																		 & \\
												 \Ttp(p/\ptp)^{R_d \gammast/g} & p < \ptp 
						\end{array} \right. \quad .
	\label{T_p}
\eeqn
Taking our gray source function $B=\sigma T^4$, Eqns. \eqnref{tau_p} and \eqnref{T_p} then gives us $B(\tau)$, which for both troposphere and stratosphere is of the form
\beqn
	B(\tau) \ \sim \  \tau^\alpha, \quad \quad \alpha \equiv \frac{4R_d\Gamma}{g\beta}\ .
	\label{B_tau}
\eeqn
This  is the key quantity in the gray radiative transfer equations \citep{pierrehumbert2010}.  \comment{Comment on importance of alpha.} These equations yield the net upward flux profile
\beqn
	F(\tau) = \Bs\exp[-(\taus-\tau)] + \int_\tau^{\taus} B(\tau')\exp[-(\tau'-\tau)] d\tau' - \int_0^\tau B(\tau')\exp[-(\tau-\tau')]d\tau' 
	\label{F_tau}
\eeqn 
The first term gives the surface contribution to the upward flux, the second the atmospheric, while the third gives the downward flux (all of which emanates from the atmosphere). This expression may be differentiated to yield the flux divergence in $\tau$ coordinates, which as shown in Appendix A \comment{construct} can be written
	\beqn
		-\ddtau{F}(\tau) = B(\tau)e^{-\tau} \ - \  [\Bs - B(\tau)]\exp[-(\taus - \tau)] \ + \ \SX + \ \AX \ 
		\label{cts_decomp}
	\eeqn
where we define `symmetric exchange' and `anti-symmetric exchange' terms \SX\ and \AX\ as 
	\beqn 
		\SX \ \equiv \   -\ \left\{ \begin{array}{cr} \int_0^\tau [B(\tau + x) - 2B(\tau) + B(\tau-x)]e^{-x}\,dx   & \tau < \taus/2 \\
																															 &  \\	
												\int_{0}^{\taus-\tau} [B(\tau + x) - 2B(\tau) + B(\tau-x)]e^{-x}\,dx   & \tau > \taus/2 
							\end{array}   \right.  
	\eeqn
and
	\beqn
		\AX  \ \equiv \   -\ \left\{ \begin{array}{cr} \int_{\tau}^{\taus-\tau}[B(\tau + x)-B(\tau)]e^{-x}	dx 	& \tau < \taus/2 \\
																																					& \\
													 \int_{\taus - \tau}^{\tau}[B(\tau - x)-B(\tau)]e^{-x} 	dx 	& \tau > \taus/2 
							\end{array}  \right. \quad .
	\eeqn
Here $\SX(\tau)$ represents exchange between level $\tau$ and a maximally thick  layer in which it is centered (in $\tau$), while \AX\ represents exchange between level $\tau$ and the remainder of the atmosphere, which will lie entirely at either greater or smaller $\tau$ values. \comment{make cartoon, expand}. 

The $B(\tau)e^{-\tau}$ term in Eqn. \eqnref{cts_decomp} is the `cooling-to-space' (CTS) term, since $e^{-\tau}$ is just the transmissivity to space at level $\tau$. The CTS approximation says that this term dominates and the others can be neglected, to a reasonable approximation. Figure \ref{cts_decomp_taus100} plots profiles of each of these terms for $\beta = 1,4$ and $\taus=100$, a typical value for the bands of interest. Rather than plot $-\pptau F$ we plot the heating rate in K/day \comment{fix sign convention}, which is also proportional to $-\ppp F$. These panels show that indeed the CTS term is a good approximation to the net heating, for both \cotwo\ in the stratosphere and \htwo\ in the troposphere.

Why does the CTS approximation work? Since heating seems to be concentrated near $\tau=1$ in Fig. \ref{cts_decomp_taus100}, let us analyze the terms in \eqnref{cts_decomp} near this level. For $\taus =100$, the ground exchange term GEX \comment{label above} is negligible. For the CTS term, we have
\beqn
\CTS = \frac{B}{e} \quad \mbox{at $\tau=1$} \ .
\eeqn
For SX, we approximate the expression in brackets as $x^2\frac{d^2 B}{d \tau^2}$ \comment{comment on this diffusion term} and invoking \eqnref{B_tau} thus obtain
\beqa
	\SX & =  & -  \frac{d^2 B}{d \tau^2}(\tau) \int_0^\tau x^2 e^{-x} dx \n \\
%			& = & \frac{d^2 B}{d \tau^2}(\tau)[2(1-e^{-\tau}) -\tau(\tau+2)e^{-\tau}] \\
			& \approx & - \frac{\alpha(\alpha-1)}{6}B \quad \mbox{at $\tau=1$} \ .
\eeqa  
For \AX\ we similarly approximate the integrand as $x\frac{d B}{d \tau}$, and so
\beqa
	\AX & =  & - \frac{d B}{d \tau}(\tau) \int_\tau^{\taus - \tau} x e^{-x} dx \n \\
%			& = & \frac{d^2 B}{d \tau^2}(\tau)[2(1-e^{-\tau}) -\tau(\tau+2)e^{-\tau}] \\
			& = & - \frac{2\alpha}{e}B \quad \mbox{at $\tau=1$} \ .
\eeqa  
For stratospheric \cotwo\ ($\beta = 1,\  \Gamma = \gammast$) and tropospheric \htwo\ ($\beta = 4,\  \Gamma = \gammatr$), we find $\alpha \approx -0.2,\ 0.2$ respectively.  Thus both the linear rate of change of $B(\tau)$ (proportional to $\alpha$) as well as its curvature (proportional to $\alpha(\alpha-1)$) are small compared to $B$ itself, and thus the CTS term dominates. Note that this domination is facilitated by partial cancellation between \SX\ and \AX, particularly for the $\beta=4$ case. \comment{comment on different shapes of profiles and emission heights? }

While this analysis explains why the CTS approximation works, it also shows how it might break down. For $\beta=1$ and $\taus < 10$, for instance, $\tau=1$ will occur in the troposphere for the  case. There the lapse rate has over 3 times its stratospheric magnitude, increasing $\alpha$ to roughly 3/4 and making \AX\ comparable to \CTS and of opposite sign. \comment{Note that \SX\ depends on curvature, which is very small for this case.} Figure \ref{cts_decomp_taus5} shows the analogs of Fig. \ref{cts_decomp_taus100}, but for $\taus=5$. Indeed we find that the CTS approximation breaks down for $\beta =1$. For $\beta=4$, however, it is still a reasonable approximation (at least away from the surface), because $\alpha$ still has its relatively low value of 0.2.  

To more systematically analyze  the validity of the CTS approximation across $(\taus,\beta)$ space, we consider the total atmospheric radiative divergence $Q \equiv \OLR-F(\taus)$ where $\OLR\equiv F(0)$. Evaluating \eqnref{F_tau} at $\tau=0$ gives
\beqn
	\OLR = \Bs e^{-\taus} + \int_0^{\taus} B(\tau') e^{-\tau'} d\tau' \ .
\eeqn
Since the second term is just the integral of the CTS term, we denote it by \Qcts. Similarly,  Evaluating \eqnref{F_tau} at $\tau=\taus$ gives
\beqn
	F(\taus) = \Bs  - \int_0^{\taus} B(\tau') e^{-(\taus-\tau')} d\tau' \ .
\eeqn
This gives
\beqn
	Q \ = \ \Qcts - (1-e^{-\taus})\Bs  + \int_0^{\taus} B(\tau') e^{-(\taus-\tau')} d\tau' \ .
\eeqn
The second and third terms are just the integral of the GEX term, which always heats the atmosphere, so we have
\beqn
	Q \ <\  \Qcts \ .
\eeqn
If the CTS approximation holds then $Q \approx \Qcts$ \comment{is the converse necessarily true?}.  We assess the degree to which this holds across $\taus-\beta$ space in Fig. \ref{Q_Qcts}, which plots $Q$ and \Qcts\ for $\beta=1,4$ and for optically thin ($\taus = 0.1$) to very optically thick ($\taus=1000$) cases. This figure shows that the CTS approximation holds for $\beta=4$ across our range of optical depths, as expected since $\alpha = 0.2$ throughout. For $\beta=1$, however, the CTS approximation breaks down for $\taus \lesssim 10$ as predicted, since at $\taus \approx 10$ the $\tau \approx 1$ level transitions from the stratosphere, where $\alpha = -1/5$, to the  troposphere where $\alpha = 3/4$.

\section{Stratospheric vs. tropospheric emission}
Given the above behavior of radiative cooling from  \cotwo\ and \htwo\ lines, how do we understand the preponderance of stratospheric emission from \cotwo\ and tropospheric emission from \htwo? Since the CTS approximation holds across the \taus\ spectrum for $\beta=4$, and since $-\ppp F = - (d\tau/dp)\pptau F$ maximizes around $\tau=1$ in this case (Figs. \ref{cts_decomp_taus100}, \ref{cts_decomp_taus5}) \comment{explain this somewhere}, tropospheric emission from \htwo\  means that active water vapor lines attain $\tau=1$ in the troposphere. Conversely,  For $\beta=1$ with $\tau=1$ in the stratosphere where $\alpha$ is small, the CTS approximation holds and $-\ppp F \sim -\pptau F$ will be significant near $\tau=1$, i.e. in the stratosphere. For  $\tau=1$ in the troposphere the situation is less straightforward, but we expect the \AX\ term to be significant in the lower troposphere (Fig. \ref{cts_decomp_taus5}) \comment{discuss this further}. Thus, a preponderance of stratospheric cooling from \cotwo\ implies that most of its active lines attain $\tau=1$ in the stratosphere. We confirm this difference between the typical $\tau=1$ levels of \htwo\ and \cotwo\ in Figure \ref{tau=1_lbl}, which shows the elevation of the $\tau=1$ level across the LW spectrum for both gases \comment{generate this LBL figure}. Why are the $\tau=1$ levels of these gases distributed so differently?

One possible answer has to do with spectroscopy and optical depths. Perhaps the \cotwo\ spectrum is populated  much more heavily by higher optical depth lines whose  whose $\tau=1$ levels then lie in the stratosphere, and conversely for the \htwo\ spectrum.  Textbook plots of \cotwo\ and \htwo\ absorption coefficients as a function of wavenumber $\kappa(k)$ \citep{pierrehumbert2010}, however, show that the largest absorption coefficients in both bands is about $10^4\ \mathrm{kg/m^2}$, and that in both  bands $\kappa(k)$ declines exponentially in $k$ from the peak. Thus, the PDF of $\kappa$ values in both these bands should be comparable. Furthermore, the path lengths of \cotwo\ and \htwo\ are comparable (in the neighborhood of $45\ \mathrm{kg/m^2}$), and thus the PDF of $\taus$ values in both bands should be comparable. These claims are confirmed in Figure \ref{taus_pdf}, which shows PDFs of \taus\ for both bands in our standard atmosphere and shows little significant difference in their shape or extent \comment{generate LBL figure, comment on continuum}.

Thus, the differing emission levels of \cotwo\ and \htwo\ are not due to differences in $\kappa$ or \taus\ distributions. Rather, they are due to the fact \cotwo\ is well-mixed ($\beta=1$) and \htwo\ is bottom-heavy ($\beta=4$). This can be seen rather straightforwardly from Eqn. \eqnref{tau_p}, which we invert to find the pressure $\pem$ at which $\tau=1$:
	\beqn
		\pem = \frac{\ps}{\taus^{1/\beta}}
	\eeqn
We plot this for $\beta=1,4$ as a function of \taus\ in Figure \ref{p1_taus}. This figure shows that when \pem\ values are plotted against a logarithmic \taus\ axis (this is appropriate because $\ln \taus$ should be distributed roughly uniformly w.r.t $k$; see previous paragraph), the majority of \pem\ values for $\beta=1$ appear in the stratosphere, whereas the majority of \pem\ values for $\beta=4$ appear in the stratosphere. 

\section{Summary and discussion}

	
%========%
% Figures    %
%========%
\pagebreak

%Figure lbl_H
\begin{figure}[h]
	\begin{center}
			\includegraphics[scale=0.5]{../plots/lbl_H.jpg}
		\caption{Spectrally-resolved cooling rates. \cotwo\ cooling is concentrated in the stratosphere, while \htwo\ cooling is concentrated in the troposphere. \comment{above figure is a placeholder}
		\label{lbl_H}
		}
	\end{center}
\end{figure}

%Figure lbl_cts
\begin{figure}[h!]
	\begin{center}
			%\includegraphics[scale=0.8]{../plots/OLR_Q.pdf}
		\caption{Spectrally-resolved CTS. This should show that CTS is a very good approximation to radiative cooling for both the \cotwo\ and \htwo\ bands. 
		\label{lbl_cts}
		}
	\end{center}
\end{figure}

%Figure cts_decomp_taus100
\begin{figure}[h!]
	\begin{center}
			\includegraphics[scale=0.6]{../plots/cts_decomp_taus100.pdf}
		\caption{Decomposition of heating rate (positive for cooling, actually) into terms in \eqnref{cts_decomp} for $\beta=1$ (left) and $\beta=4$ (right) for $\taus=100$. Note that in both cases the CTS approximation is accurate.
		\label{cts_decomp_taus100}
		}
	\end{center}
\end{figure}

%Figure cts_decomp_taus5
\begin{figure}[h!]
	\begin{center}
			\includegraphics[scale=0.6]{../plots/cts_decomp_taus5.pdf}
		\caption{As in Fig. \ref{cts_decomp_taus100}, but for $\taus=5$. The CTS approximation continues to hold for $\beta=4$, but breaks down for $\beta=1$.
		\label{cts_decomp_taus5}
		}
	\end{center}
\end{figure}


%Figure Q_Qcts
\begin{figure}[h!]
	\begin{center}
			\includegraphics[scale=0.8]{../plots/Q_Qcts.pdf}
		\caption{Plots of column-integrated cooling $Q$, as well as column-integrated cooling-to-space \Qcts. For $\beta=4$ we have $Q\approx \Qcts$ for all \taus, but for $\beta=1$ the CTS approximation breaks down when $\taus \lesssim 10$. 
		\label{Q_Qcts}
		}
	\end{center}
\end{figure}

%Figure tau=1_lbl
\begin{figure}[h!]
	\begin{center}
			%\includegraphics[scale=0.8]{../plots/p1_taus.pdf}
		\caption{Spectrally-resolved plot of elevation of $\tau=1$ for lines in the \cotwo\ and \htwo\ bands. This should show that $\tau=1$ occurs primarily in the stratosphere for the \cotwo\ band and primarily in the troposphere for the \htwo\ band.
		\label{tau=1_lbl}
		}
	\end{center}
\end{figure}

%Figure taus_pdf
\begin{figure}[h!]
	\begin{center}
			%\includegraphics[scale=0.8]{../plots/p1_taus.pdf}
		\caption{PDF \taus\ values for \htwo\ and \cotwo\ band. These should show similar distributions for the two bands.
		\label{taus_pdf}
		}
	\end{center}
\end{figure}

%Figure p1_taus
\begin{figure}[h!]
	\begin{center}
			\includegraphics[scale=0.8]{../plots/p1_taus.pdf}
		\caption{For $\beta=1$, the majority of \pem\ values (when distributed in $\ln \taus$) are stratospheric, whereas for $\beta=4$ the majority of \pem\ values are tropospheric.
		\label{p1_taus}
		}
	\end{center}
\end{figure}




\pagebreak

\bibliographystyle{apa}
\bibliography{/Users/nadir/Dropbox/bibtex_mendeley/library}


\end{document}

