%% Draft settings
\documentclass[10pt]{article}
\usepackage{amsmath}
\usepackage{amssymb}
\usepackage{graphicx}
\usepackage{subfigure}
\usepackage{color}
 \usepackage{lineno}
\usepackage{simplemargins}
\usepackage{natbib}

% \linenumbers*[1]
 \usepackage[T1]{fontenc} % For citing barki{\dj}ija
\setkeys{Gin}{draft=false}


% Margins
\setleftmargin{1in}
\setrightmargin{1in}
\setbottommargin{1in}
\settopmargin{1in}
 
% My Commands
%\input{tensor_book_shrtcts.tex}
\newcommand{\comment}[1]{\textcolor{blue}{[{#1}]}}


%%Nadir's Shortcuts
\newcommand{\beqn}{\begin{equation}}
\newcommand{\eeqn}{\end{equation}}
\newcommand{\beqa}{\begin{eqnarray}}
\newcommand{\eeqa}{\end{eqnarray}}
\newcommand{\beqanonum}{\begin{eqnarray*}}
\newcommand{\eeqanonum}{\end{eqnarray*}}
\newcommand{\beqnonum}{\begin{equation*}}
\newcommand{\eeqnonum}{\end{equation*}}
\newcommand{\jump}{\vspace{0.5cm}}
\newcommand{\bbf}{\begin{bf}}
\newcommand{\ebf}{\end{bf}}
\newcommand{\eqnref}[1]{(\ref{#1})}
\newcommand{\defn}[1]{\begin{bf}\emph{#1}\end{bf}}
\newcommand{\reals}{\ensuremath{\mathbb{R}}}
\newcommand{\complex}{\ensuremath{\mathbb{C}}}
\newcommand{\integers}{\ensuremath{\mathbb{Z}}}
\newcommand{\half}{\ensuremath{\frac{1}{2}}}
\newcommand{\n}{\nonumber}
\newcommand{\inverse}{^{-1}}

%calculus shorthand
\newcommand{\timeder}{\frac{d}{dt}}
\newcommand{\partialder}[1]{\frac{\partial}{\partial #1}}
\newcommand{\partialderf}[2]{\ensuremath{\frac{\partial #1}{\partial #2}}}
\newcommand{\der}[2]{\ensuremath{\frac{d #1}{d #2}}}
\newcommand{\dx}{\ensuremath{\frac{d}{dx}}}
\newcommand{\ddx}{\ensuremath{\frac{d}{dx}}}
\newcommand{\kvec}{\ensuremath{\vec{k}}}
\newcommand{\uvec}{\ensuremath{\mathbf{u}}}
\newcommand{\zhat}{\ensuremath{\mathbf{\hat{z}}}}
\newcommand{\khat}{\ensuremath{\mathbf{\hat{k}}}}
\newcommand{\unitvect}[1]{\ensuremath{\mathbf{\hat{#1}}}}
\newcommand{\ppx}{\ensuremath{\partial_x}}
\newcommand{\ppy}{\ensuremath{\partial_y}}
\newcommand{\ppz}{\ensuremath{\partial_z}}
\newcommand{\ppt}{\ensuremath{\partial_T}}
\newcommand{\ppp}{\ensuremath{\partial_p}}

% units shorthand
\newcommand{\Wmsq}{\ensuremath{\mathrm{W/m^2}}}
\newcommand{\Kinverse}{\ensuremath{\mathrm{K^{-1}}}}

% radiation shorthand
\newcommand{\cotwo}{\ensuremath{\mathrm{CO_2}}}
\newcommand{\othree}{\ensuremath{\mathrm{O_3}}}
\newcommand{\htwo}{\ensuremath{\mathrm{H_2O}}}
\newcommand{\QLW}{\ensuremath{Q}}
\newcommand{\QLWgray}{\ensuremath{Q_\mathrm{gray}}}
\newcommand{\QSW}{\ensuremath{Q_\mathrm{SW}}}
\newcommand{\Qnet}{\ensuremath{Q_\mathrm{net}}}
\newcommand{\FLW}{\ensuremath{F^\mathrm{LW}}}
\newcommand{\FSW}{\ensuremath{F^\mathrm{SW}}}
\newcommand{\USW}{\ensuremath{U^\mathrm{SW}}}
\newcommand{\DSW}{\ensuremath{D^\mathrm{SW}}}
\newcommand{\Fnet}{\ensuremath{F^\mathrm{net}}}
\newcommand{\olr}{\ensuremath{\mathrm{OLR}}}
\newcommand{\OLR}{\ensuremath{\mathrm{OLR}}}
\newcommand{\OLRgray}{\ensuremath{\mathrm{OLR_{gray}}}}
\newcommand{\trans}{\ensuremath{\mathcal{T}}}
\newcommand{\solar}{\ensuremath{I_0}}
\newcommand{\cool}{\ensuremath{\mathcal{C}}}
\newcommand{\cminverse}{\ensuremath{\mathrm{cm^{-1}}}}
\newcommand{\pierre}{P10}
\newcommand{\tauk}{\ensuremath{\tau_k}}
\newcommand{\taus}{\ensuremath{\tau_s}}
\newcommand{\Tone}{\ensuremath{T_{\tau=1}}}

% meteorology shorthand
\newcommand{\qv}{\ensuremath{q}}
\newcommand{\rhov}{\ensuremath{\rho_\mathrm{v}}}
\newcommand{\qvstar}{\ensuremath{q^*}}
\newcommand{\Ts}{\ensuremath{T_\mathrm{s}}}
\newcommand{\ps}{\ensuremath{p_s}}
\newcommand{\RH}{\ensuremath{\mathrm{RH}}}
\newcommand{\WVP}{\ensuremath{\mathrm{WVP}}}

%Variables
\newcommand{\figurepath}{../figures/}


\begin{document}

%% ------------------------------------------------------------------------ %%
%
%  TITLE
%
%% ------------------------------------------------------------------------ %%


\title{Relative Humidity, Radiative Cooling, and the Gray Approximation}

%% ------------------------------------------------------------------------ %%
%
%  AUTHORS AND AFFILIATIONS
%
%% ------------------------------------------------------------------------ %%


 \author{Nadir Jeevanjee\footnote{Department of Physics, University of California at Berkeley, Berkeley, CA 94702  USA. jeevanje@berkeley.edu (corresponding author)} \footnote{Climate and Ecosystems Science Division, Lawrence Berkeley National Laboratory, Berkeley, CA USA} and David Romps\footnote{Department of Earth and Planetary Sciences, University of California at Berkeley, Berkeley, CA 94702  USA.} \footnote{Climate and Ecosystems Science Division, Lawrence Berkeley National Laboratory, Berkeley, CA USA}
}

\maketitle

\begin{abstract}
We show here that atmospheric radiative cooling typically decreases as relative humidity (RH) decreases. This contrasts the behavior of outgoing longwave radiation as a function of RH, as well as expectations from the gray approximation. This simultaneous decrease of cooling and increase in OLR with drying occurs whenever surface emission to space is non-negligible. The gray approximation, tuned appropriately, can capture this behavior, at least qualitatively.


%\vspace{0.5cm}
%
%
\end{abstract}


%% ------------------------------------------------------------------------ %%
%
%  TEXT
%
%% ------------------------------------------------------------------------ %%


\section {Introduction}
Though the increase of outgoing longwave radiation (OLR) with decreased relative humidity (RH) is well-known \citep{pierrehumbert2010}, the dependence of longwave atmospheric radiative cooling on RH has not been as well studied. This dependence has been implicated, however, as a key feedback in convective self-aggregation \citep[e.g.][]{muller2015, emanuel2014,wing2014}, and so the time seems ripe for a closer examination. 

Our current understanding of how LW radiative cooling depends on RH is quite murky.  For instance, since the cooling-to-space (CTS) approximation says that radiative cooling profiles can be approximated to first order by cooling-to-space profiles \citep[e.g]{jeevanjee2016,clough1992,rodgers1966}, where the cooling-to-space of a given layer is simply its contribution to the OLR, one might expect a close connection between column-integrated longwave cooling \QLW\ and OLR. In particular, one might expect that both quantities behave similarly under perturbations with respect to, say, surface temperature \Ts\ or RH. Indeed, under increases in \Ts\  both OLR and \QLW\ increase, as is well-known \citep{jeevanjee2016,pendergrass2014,ogorman2012,stephens2008}. Should OLR and \QLW\  also behave similarly under changes in RH? More specifically, since we know that  OLR increase with decreased RH,  should we expect \QLW\  to also increase with drying?

Attempts to answer this question further expose the murkiness of our understanding. If we take a gray view of the atmosphere, in which radiative transfer occurs through water vapor but at a single frequency with absorption coefficient $\kappa$, then we might think of all OLR as being generated by thermal emission at the level of unit optical depth. We describe this level by its temperature \Tone  (and in general use temperature as a vertical coordinate throughout this study).  Drying the atmosphere causes this  level to descend, increasing \Tone\ and hence increasing OLR. By the CTS approximation, \QLW\ should then increase as well.

A spectrally resolved view, however, potentially gives a different result. In this view, it is well-known that some wavenumbers $k$ have water-vapor optical depth  $\tauk(\Ts) \gtrsim 1$ and thus have significant cooling-to-space, whilst others have $\tauk(\Ts) \ll 1$ and thus do not significantly contribute to atmospheric cooling (these latter wavenumbers constitute the water-vapor `window'). If RH decreases, this pushes some  wavenumbers into the $\tauk(\Ts) \ll 1$ regime, eliminating their contribution to atmospheric cooling and \emph{reducing} \QLW.

The goal of this short paper is to resolve the tension between the two arguments presented above,  and to clarify the relationship between OLR and \QLW. We will use a comprehensive radiation scheme, RRTM \citep{mlawer1997}, for benchmark calculations which will comprise our main result. We then ask whether a gray radiation model can at least qualitatively match the benchmark's behaviour, and finally employ an analytical but spectrally-resolved model to try and understand our benchmark results. 

\section{Simulations}
Methods here	


%======================%
% Content 1    %
%======================%

\section{Content 1}
We begin by calculating \QLW\ and OLR for \Ts=300 K thermodynamic profiles, with specific humidity \qv\ scaled by factors of (1, 0.8, 0.4, 0.6, 0.2). The results are shown in Fig. \ref{fig_olr_qlw_window}a, which clearly show an increase in OLR and a \emph{decrease} in \QLW\ with drying. 

This result is at odds with the gray thinking outlined above, in which drying leads to an increase in \Tone\ and hence an increase in cooling-to-space and hence \QLW. But is it in fact true that a gray model cannot qualitatively reproduce the behavior of Fig. \ref{fig_olr_qlw_window}a? Consider a gray model defined by 
	\begin{subequations}
	\begin{align}
			\der{U}{z} & = - \kappa \rhov (U-\sigma T^4) \\
			  		& \n \\
			\der{D}{z} & = \kappa \rhov (D-\sigma T^4) 
	\end{align}			
	\label{eqn_gray_model}
	\end{subequations}
and the boundary conditions $U = \sigma \Ts^4$ at model bottom and $D=0$ at model top. Setting $\kappa$ to 0.12 $\mathrm{kg/m^2}$ yields a gray OLR equal to the RRTM OLR for unscaled \qv, and we then compute gray cooling and OLR using the same thermodynamic profiles with modified \qv\ profiles as above. The results are shown in Fig. \ref{fig_olr_qlw_window}a, and show that indeed the gray model can exhibit a simultaneous increase in OLR and decrease in \QLW, qualitatively similar to RRTM.

To understand this, we plot $-\ppt F$ profiles for both RRTM and our gray model in Fig. \ref{fig_olr_qlw_window}b,d. Though a prominent emission peak at $T=\Tone$ is evident for each gray $-\ppt F$ profile, and this peak does descend with drying, each peak also has an appreciable width, more and more of which is `submerged' below the surface as RH decreases, causing the eventual decline in $\QLW = \int - \ppt F$.

If the emission peak starts to get cut-off by the surface in this way, it means that the optically thick condition  $\tau \gg 1$ ceases to hold, as confirmed by the $\taus\equiv \tau(\Ts)$ values given in Fig. \ref{fig_olr_qlw_window}d. An unavoidable consequence of this is that the surface emission $\sigma\Ts^4 \exp(-\taus)$ must increase; indeed, over this range of \taus,   $\sigma\Ts^4 \exp(-\taus)$ increases from 1 \Wmsq\ to 140 \Wmsq. One can think of this as a `widening' of the water-vapor window, in a gray setting.

% make the connection to a widening water vapor window.
The foregoing suggests that the naive gray view mentioned in the introduction is incomplete, because it fails to account for the finite vertical depth of the emission peak at $T=\Tone$, which allows for partial elimination of emission as \Tone\ increases with drying. The foregoing also suggests that the spectral view, in which cooling declines with drying because the water vapor window widens and some atmospheric emission is eliminated, is correct, at least insofar as that same physics manifests in a gray radiation model. 

If this spectral view is correct, then once the water vapor window closes, in either RRTM or our gray model, we should expect to see an \emph{increase} in cooling with drying, due to the increase in \Tone (which would occur independently for each line individually in the spectrally-resolved RRTM calculation). To check this, we perform the same computation as above, but for \Ts=330 K, well above the 310 K at which the water-vapor window closes \citep{goldblatt2013}. 


	
%================%
% Content 2    %
%================%
\section{Content 2}


%=================================%
% Content 3   %
%=================================%
\section{Content 3 }


%=================%
% Content 4    %
%=================%
\section{Content 4 }


%===========%
% Discussion    %		
%===========%
\section{Summary and discussion}
Our results can be summarized as follows:
\begin{itemize}
	\item 
	\item
	\item
\end{itemize}


%===========%
% Appendix       %		
%===========%

\section*{Appendix}
\appendix


	%==================%
	% appendix_1  %
	%===================%	
	\section{Appendix 1 title}	 \label{appendix_1}	


	%==================%
	% appendix_2  %
	%===================%
	\section{Appendix 2 title} \label{appendix_2}
	
	
%========%
% Figures    %
%========%
\pagebreak

% fig_olr_qlw_window
\begin{figure}[h]
	\begin{center}
			\includegraphics[scale=0.7]{../plots/olr_qlw_window.pdf}
		\caption{Caption here
		\label{fig_olr_qlw_window}
		}
	\end{center}
\end{figure}

%% fig_gray
%\begin{figure}[h]
%	\begin{center}
%			\includegraphics[scale=0.7]{../plots/gray.pdf}
%		\caption{Caption here
%		\label{fig_gray}
%		}
%	\end{center}
%\end{figure}




\bibliographystyle{apa}
\bibliography{/Users/climateloaner/Dropbox/bibtex_mendeley/library}
%\bibliography{/Users/nadir/Dropbox/bibtex_mendeley/library}


\end{document}

